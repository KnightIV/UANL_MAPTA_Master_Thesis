\chapter{Modelo Computacional} \label{metodologia:modelocomputacional}

Usando todas las curvas de luz disponible para el sistema \atoObjId se puede
generar un modelo computacional cuyos parámetros físicos resultan en una curva
de luz sintética que explique de manera adecuada las curvas fotométricas
observadas. Este método al final daría como resultado una \textit{solución
fotométrica} del sistema, en el cual se reportan los valores óptimos de cada
parámetro y la incertidumbre dada por la calidad de los datos. A continuación se
plasma el proceso que se llevó a cabo para llegar a una solución fotométrica del
sistema \atoObjIdNoSpace; debido a la alta dimensionalidad del problema de
sistemas binarios estelares, no se puede garantizar que esta sea la única
combinación de parámetros que mejor ajusten el modelo a los datos, aunque esta
posibilidad es mitigada utilizando las herramientas disponibles en PHOEBE.

\section{Estimaciones Iniciales}
\label{metodologia:modelocomputacional:estimacionesiniciales}

Una vez determinado el periodo orbital del sistema se puede empezar un estudio
de la morfología de las curvas de luz en fase. PHOEBE para facilitar esto ofrece
distintos métodos para generar los las primeras estimaciones de parámetros
físicos del sistema. El estimador \textbf{EBAI-KNN} para estimar los siguientes
parámetros: el \textit{tiempo de conjunción superior} (\code{t0\_supconj}), la
\textit{razón de temperaturas} (\code{teffratio}), la \textit{inclinación
	orbital} (\code{incl@binary}), el \textit{factor de relleno}
(\textit{fillout factor} en inglés, \code{fillout\_factor}), y la
\textit{razón de masas} (\code{q}). A pesar que dentro de PHOEBE estén
implementados estimadores adicionales, solo se puede aplicar el
\textbf{EBAI-KNN} estimador; esto se debe a que el modelo del sistema del que
parte este trabajo corresponde al de una binaria en contacto (elegido por la
morfología aparente de la curva de luz de Iturbide).

Dentro del Jupyter Notebook
\href{https://github.com/KnightIV/UANL_MAPTA_PlanObservaciones/blob/main/analisis/phoebe_model/estimations/ebai-default.ipynb}{\code{ebai-default.ipynb}}
se puede encontrar el código con el que se llevaron a cabo las pruebas de
estimación de parámetros. El estimador \textbf{EBAI-KNN} puede que obtenga
diferentes soluciones del sistema dependiendo de la curva de luz utilizada; por
lo cual se esperaba que obtuviera diferentes resultados dependiendo de la curva
de entrada. Para obtener un panorama completo de las posibles soluciones
fotométricas so ejecutaron varios estimadores de PHOEBE, cada uno operando sobre
una diferente combinación de curvas de luz; se corrió un estimador por cada
curva de luz individual, al igual que unos estimadores que tuvieron de entrada
una combinación de curvas de luz de Gaia e Iturbide. El experimento completo
junto a sus curvas de luz sintéticas correspondientes se pueden ver en el
Notebook antedicho, acompañado de las gráficas resultantes de cada estimador.

\subsection{Elección del Modelo Inicial}
\label{metodologia:modelocomputacional:estimacionesiniciales:eligiendomodeloinicial}

Una consideración importante en el proceso de modelación computacional es la
existencia de diferentes soluciones fotométricas dado un mismo conjunto de
datos. Esto se debe a la ortogonalidad de los parámetros en el sistema; dos o
más parámetros pueden estar en un estado de degeneración, donde existe una
relación lineal entre estos, lo cual significa que no existe una solución única
correcta del sistema. Para decidir entre los varios estimadores se tomó como
criterio de selección el ajuste del "forward model" a los datos mediante la
estadística $\chi^2$. Estos se pueden ver en la \reffigure{chiSqrdFigure}.
Viendo solo la medida de ajuste $\chi^2$ total para cada modelo se llegaría a la
conclusión que el modelo \code{ebai\_knn\_iturbide\_aviles\_raw\_solution} es el
que más coincide con los datos observacionales. Sin embargo, es necesario ver el
ajuste a cada curva de luz individual; la curva de luz obtenida de Iturbide
tiene mayor dispersión y puntos de valor que cualquiera de las otras curvas. Es
por esta razón que el $\chi^2$ de \code{lc\_iturbide\_aviles\_raw} es varias
ordenes de magnitud mayor que cualquier otro estimador. Por lo tanto se eligió
las estimaciones de \code{ebai\_knn\_gaia\_norm\_solution}, el cual utilizó la
fotometría de Gaia normalizada. Los parámetros de este estimador fueron
idénticos al estimador que utilizó los datos directos de la base de datos de
Gaia. El resultado inicial del modelo se puede ver en la
\reffigure{ebaiKnnGaiaEstimateModel}, junto a los parámetros del modelo en la
\reftable{ebaiKnnInitialEstimationsValues}.

\begin{figure}[!ht]
	\centering
	\includegraphics[scale=0.4]{Metodologia/Secciones/ModeloComputacional/Figures/EstimadoresChiResultados.png}

	\caption{Resultados del ajuste ($\chi^2$) de los modelos sintéticos
		generados utilizando los parámetros de los estimadores. Cada estadística fue
		calculada con respecto a todos los datos observacionales disponibles, sin
		importar las combinaciones de curvas de luz utilizadas para hacer la
		estimación. \code{raw\_model} corresponde al modelo inicial que ofrece
		PHOEBE a través de la función \code{phoebe.default\_contact\_binary()}. Los
		nombres de los estimadores en el eje vertical de la gráfica indican los
		datos observacionales utilizados para cada estimación de parámetros.}
	\label{chiSqrdFigure}
\end{figure}

\begin{figure}[!ht]
	\centering
	\xincludegraphics[scale=0.5]{Metodologia/Secciones/ModeloComputacional/Figures/ebaiKnnIturbideNorm.png}
	\xincludegraphics[scale=0.5]{Metodologia/Secciones/ModeloComputacional/Figures/ebaiKnnGaiaRaw.png}
	\xincludegraphics[scale=0.5]{Metodologia/Secciones/ModeloComputacional/Figures/ebaiKnnZtf.png}

	\caption{Modelos sintéticos del modelo utilizando los parámetros estimados
	por \code{ebai\_knn\_gaia\_norm\_solver} junto a los residuos en los flujos para
	cada curva de luz. Estos modelos fueron sintetizados utilizando un factor de
	escala de flujos flexible, utilizando la opción \code{pblum\_mode =
	"dataset\_scaled"}, el cual nos permite analizar la morfología del modelo
	sintético sin considerar por ahora el efecto en la escala de la curva de
	parámetros relacionados con la luminosidad de cada componente, como las
	temperaturas absolutas de ambas estrellas. Estos parámetros son ajustados en
	los siguientes pasos de afinación del modelo.}
	\label{ebaiKnnGaiaEstimateModel}
\end{figure}

% TODO: style table
\begin{table}[!ht]
	\centering
	\begin{tabular}{|l|l|}
		\hline
		% \rowcolor{blue}
		\thead{Parámetro}                        & \thead{Valor} \\
		\hline
		\code{t0\_supconj@binary}                & -0.03748 d    \\
		\hline
		\code{teffratio@binary}                  & 0.98573       \\
		\hline
		\code{incl@binary}                       & 71.07442 deg  \\
		\hline
		\code{fillout\_factor@contact\_envelope} & 0.18522       \\
		\hline
		\code{q@binary}                          & 1.85648       \\
		\hline
	\end{tabular}
	\caption{Resultados adoptados de las estimaciones iniciales, utilizando el
		estimador cuyos datos de entrada fueron solo datos de Gaia. Las unidades
		de cada valor son especificadas excepto para los parámetros
		adimensionales.}
	\label{ebaiKnnInitialEstimationsValues}
\end{table}

\section{Optimización de Parámetros}

Como se puede ver en la \reffigure{ebaiKnnGaiaEstimateModel} el modelo inicial
de PHOEBE no se ajusta perfectamente bien a los dados datos observacionales. Los
residuos de los flujos no llegan a estar planos alrededor de 0; se alcanza a
apreciar un comportamiento oscilatorio en fase. Por lo tanto es necesario
ajustar estos parámetros para llegar al mínimo global en el espacio de
parámetros. PHOEBE ofrece varias herramientas partiendo de una estimación
inicial, haciendo uso de métodos numéricos para evaluar el espacio de parámetros
y llegar a un modelo con el mejor ajuste a los datos observacionales. 