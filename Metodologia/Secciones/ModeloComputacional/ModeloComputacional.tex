\section{Modelo Computacional} \label{metodologia:modelocomputacional}

Usando todas las curvas de luz disponible para el sistema \atoObjId --- tanto de
Gaia como los datos recabados de Iturbide --- se puede generar un modelo
computacional cuyas propiedades físicas pueden adecuadamente explicar los datos
observacionales. Este método al final daría como resultado una \textit{solución
fotométrica} del sistema. En el mejor de los casos, esta solución muestra un
valor satisfactorio de ajuste a los datos observacionales. A continuación se
plasma el proceso que se llevó a cabo para llegar a una solución fotométrica del
sistema \atoObjIdNoSpace; esta solución no es única en el sentido que otra
combinación de parámetros podría llegar a las mismas conclusiones.

\subsection{Estimaciones Iniciales} \label{metodologia:modelocomputacional:estimacionesiniciales}

Una vez determinado el periodo orbital del sistema se puede empezar un estudio
de la morfología de las curvas de luz en fase. PHOEBE para facilitar esto ofrece
distintos métodos para generar los las primeras estimaciones de parámetros
físicos del sistema. El estimador \textbf{EBAI-KNN} para estimar los siguientes
parámetros: el \textit{tiempo de conjunción superior} (\code{t0\_supconj}), la
\textit{razón de temperaturas} (\code{teffratio}), la \textit{inclinación
orbital} (\code{incl@binary}), el \textit{factor de relleno}
(\quotes{\textit{fillout factor}}, \code{fillout\_factor}), y la \textit{razón
de masas} (\code{q}). A pesar que dentro de PHOEBE estén implementados
estimadores adicionales, solo se puede aplicar el \textbf{EBAI-KNN} estimador;
esto se debe a que el modelo del sistema inicial corresponde al de una binaria
en contacto (indicado por su morfología).

Dentro del Jupyter Notebook
\href{https://github.com/KnightIV/UANL_MAPTA_PlanObservaciones/blob/main/analisis/phoebe_model/estimations/ebai-default.ipynb}{\code{ebai-default.ipynb}}
se puede encontrar el código con el que se llevaron a cabo las pruebas de
estimación de parámetros. El estimador \textbf{EBAI-KNN} puede obtener
diferentes soluciones dependiendo de la curva de luz utilizada; por lo cual se
esperaba que obtuviera diferentes resultados dependiendo de la curva de entrada.

% \begin{table}
% 	\centering
% 	\begin{tabular}{|l|l|l|l|l|l|l|}
% 		\hline
% 		  & \code{t0\_supconj} & \code{teffratio} & \code{incl@binary} & \code{fillout\_factor} & \code{q} \\ \hline
% 		\thead{Iturbide}               & 1 & 2 \\ \hline
% 		\thead{Iturbide (Normalizada)} & 3 & 4 \\ \hline
% 		\thead{Gaia G}                       &  & 5 \\ \hline
% 		\thead{Gaia G (Normalizada)} \\ \hline
% 		\thead{Gaia G\textsubscript{RP}} &  6 & \\ \hline
% 		\thead{Gai G\textsubscript{RP} (Normalizada)} \\ \hline

% 	\end{tabular}
% 	\caption{EBAI KNN Estimaciones}
% \end{table}