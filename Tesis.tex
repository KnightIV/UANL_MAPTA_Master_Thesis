% \documentclass[12pt,a4paper]{report}
\documentclass{sty/uanl}

\usepackage[utf8]{inputenc}

% \usepackage{geometry}

\usepackage{makecell}
\usepackage[T1]{fontenc}
\renewcommand\theadfont{\bfseries\sffamily}

\usepackage{float}
\usepackage{titlesec}
\usepackage{amssymb,amsmath,mathtools}
\usepackage[version=4]{mhchem}
\usepackage[]{graphicx} \graphicspath{ {Fig/} }
\usepackage{caption}
\usepackage[]{xcolor}
\usepackage{booktabs, multirow}
\usepackage[normalem]{ulem}
% \usepackage[colorlinks,linkcolor=MyDarkBlue,urlcolor=blue,citecolor=blue]{hyperref}
\usepackage{hyperref}
    \hypersetup{breaklinks=true,colorlinks=true,linkcolor=black,citecolor=black,urlcolor=black}
% \usepackage{hyperref}
\usepackage[nottoc]{tocbibind}
\usepackage{url}
% \usepackage[spanish,es-nodecimaldot,es-tabla]{babel}
\usepackage[group-digits=true,group-minimum-digits=4]{siunitx}
\usepackage{xcoffins}
\usepackage{annotate-equations}
\usepackage{mathtools}
\usepackage[euler]{textgreek}
\usepackage{titling}
\usepackage{lstautogobble}

% centering table column values: https://tex.stackexchange.com/questions/157389/how-to-center-column-values-in-a-table#157400 
\usepackage{array}
\newcolumntype{P}[1]{>{\centering\arraybackslash}p{#1}}

\usepackage{manfnt}
\usepackage{csquotes}
\usepackage[style=apa]{biblatex}

\usepackage{layout}

\DeclareDelimFormat[textcite]{finalnamedelim}{\addspace\&\space}

\addbibresource{ref.bib}

\ExplSyntaxOn
\NewCoffin\imagecoffin
\NewCoffin\labelcoffin

%region Labeled individual graphs https://tex.stackexchange.com/questions/454660/adding-a-label-on-top-of-figure
\keys_define:nn { miguel/label }
 {
  label   .tl_set:N = \l_miguel_label_tl,
  labelbox .bool_set:N = \l_miguel_label_box_bool,
  labelbox .default:n = true,
  fontsize .tl_set:N = \l_miguel_label_size_tl,
  fontsize .initial:n = \footnotesize,
  pos .choice:,
  pos/nw .code:n = \tl_set:Nn \l_miguel_label_pos_tl { left,up },
  pos/ne .code:n = \tl_set:Nn \l_miguel_label_pos_tl { right,up },
  pos/sw .code:n = \tl_set:Nn \l_miguel_label_pos_tl { left,down },
  pos/se .code:n = \tl_set:Nn \l_miguel_label_pos_tl { right,down },
  pos/n .code:n = \tl_set:Nn \l_miguel_label_pos_tl { hc,up },
  pos/w .code:n = \tl_set:Nn \l_miguel_label_pos_tl { left,vc },
  pos/s .code:n = \tl_set:Nn \l_miguel_label_pos_tl { hc,down },
  pos/e .code:n = \tl_set:Nn \l_miguel_label_pos_tl { right,vc },
  pos .initial:n = nw,
  unknown .code:n   = \clist_put_right:Nx \l_miguel_label_clist
                       { \l_keys_key_tl = \exp_not:n { #1 } }
 }
\clist_new:N \l_miguel_label_clist
\box_new:N \l_miguel_label_box
\box_new:N \l_miguel_label_image_box

\NewDocumentCommand{\xincludegraphics}{O{}m}
 {
  \group_begin:
  \tl_clear:N \l_miguel_label_tl
  \clist_clear:N \l_miguel_label_clist
  \keys_set:nn { miguel/label } { #1 }
  \tl_if_empty:NTF \l_miguel_label_tl
   {
    \miguel_includegraphics:Vn \l_miguel_label_clist { #2 }
   }
   {
    \SetHorizontalCoffin\imagecoffin
     {
      \miguel_includegraphics:Vn \l_miguel_label_clist { #2 }
     }
    \SetHorizontalCoffin\labelcoffin
     {
      \raisebox{\depth}
       {
        \bool_if:NTF \l_miguel_label_box_bool
         { \fcolorbox{white}{white}{\l_miguel_label_size_tl\l_miguel_label_tl} }
         { \l_miguel_label_size_tl\l_miguel_label_tl }
       }
     }
    \SetVerticalPole\imagecoffin{left}{3pt+\CoffinWidth\labelcoffin/2}
    \SetVerticalPole\imagecoffin{right}{\Width-3pt-\CoffinWidth\labelcoffin/2}
    \SetHorizontalPole\imagecoffin{up}{\Height-3pt-\CoffinHeight\labelcoffin/2}
    \SetHorizontalPole\imagecoffin{down}{3pt+\CoffinHeight\labelcoffin/2}
    \use:x{\JoinCoffins\imagecoffin[\l_miguel_label_pos_tl]\labelcoffin[vc,hc]} 
    \TypesetCoffin\imagecoffin
   }
   \group_end:
 }
\NewDocumentCommand{\setlabel}{m}
 {
  \keys_set:nn { miguel/label } { #1 }
 }

\cs_new_protected:Nn \miguel_includegraphics:nn
 {
  \includegraphics[#1]{#2}
 }
\cs_generate_variant:Nn \miguel_includegraphics:nn { V }
%endregion

\ExplSyntaxOff

% region absolute magnitude symbols https://tex.stackexchange.com/questions/43008/absolute-value-symbols#43009
\DeclarePairedDelimiter\abs{\lvert}{\rvert}%
\DeclarePairedDelimiter\norm{\lVert}{\rVert}%

% Swap the definition of \abs* and \norm*, so that \abs
% and \norm resizes the size of the brackets, and the 
% starred version does not.
\makeatletter
\let\oldabs\abs
\def\abs{\@ifstar{\oldabs}{\oldabs*}}
%
\let\oldnorm\norm
\def\norm{\@ifstar{\oldnorm}{\oldnorm*}}
\makeatother
% endregion

\sisetup{
    text-series-to-math = true ,
    propagate-math-font = true
}

% SQL formatting code
    % Taken from https://tex.stackexchange.com/questions/455993/formatting-sql-code
\usepackage{xcolor,listings}
\usepackage{textcomp}
\usepackage{color}

\definecolor{codegreen}{rgb}{0,0.6,0}
\definecolor{codegray}{rgb}{0.5,0.5,0.5}
\definecolor{codeblue}{rgb}{0, 0.2, 0.8}
\definecolor{codegold}{RGB}{198, 152, 82}
\definecolor{backcolour}{rgb}{0.97, 0.97, 0.97}

\definecolor{bookColor}{cmyk}{0, 0, 0, 0.90}
\color{bookColor}

\lstset{upquote=true}

\lstdefinestyle{mystyle}{
    backgroundcolor=\color{backcolour},   
    commentstyle=\color{codegreen},
    keywordstyle=\color{codeblue},
    numberstyle=\numberstyle,
    stringstyle=\color{codegold},
    % basicstyle=\footnotesize\ttfamily,
    basicstyle=\linespread{1.2}\footnotesize\ttfamily,
    breakatwhitespace=true,
    breaklines=true,
    captionpos=b,
    keepspaces=true,
    numbers=left,
    numbersep=10pt,
    numbersep=5pt,
    xleftmargin=3em,
    frame=single,
    framexleftmargin=3em,
    showspaces=false,
    showstringspaces=false,
    showtabs=false,
    tabsize=2
}
\lstset{style=mystyle}

\newcommand\numberstyle[1]{%
    \footnotesize
    \color{codegray}%
    \ttfamily
    \ifnum#1<10 0\fi#1 |%
}

%------------------------------------------------------------------------------------
\newcommand{\eg}{\`e}
\newcommand{\ea}{\'e}
\newcommand{\Eg}{\`E}
\newcommand{\og}{\`o}
\newcommand{\ag}{\`a}
\newcommand{\ug}{\`u}
\newcommand{\ig}{\`{\i}}
\newcommand{\R}{\textsf{R}}
\newcommand{\Fortran}{\textsf{Fortran}}

\newcommand{\gaiaNoSpace}{\textit{Gaia}}
\newcommand{\atoObjIdNoSpace}{ATO J339.9469+45.1464}
\newcommand{\atoObjId}{\atoObjIdNoSpace \space}
\newcommand{\compstarIdNoSpace}{TYC 3620-332-1}
\newcommand{\compstarId}{\compstarIdNoSpace \space}
\newcommand{\gaia}{\gaiaNoSpace \space}
\newcommand{\quotes}[1]{``#1''}
\newcommand{\code}[1]{\texttt{#1}}
\newcommand{\Angstrom}{\textup{~\AA}}

\newcommand{\citeyearparen}[1]{\textcite{#1}}
% \newcommand{\citetbookchapter}[2]{\textcite[\textcolor{MyLightBlue}{Capítulo #2}]{#1}}
% \newcommand{\citetbooksection}[2]{\textcite[\textcolor{MyLightBlue}{Sección #2}]{#1}}
\newcommand{\citetbookchapter}[2]{\textcite[Capítulo #2]{#1}}
\newcommand{\citetbooksection}[2]{\textcite[Sección #2]{#1}}

% https://tex.stackexchange.com/questions/11031/how-to-do-the-curvy-l-for-lagrangian-or-laplace-transforms#11144
\newcommand{\Lagr}{\mathcal{L}}
%------------------------------------------------------------------------------------
\definecolor{MyGray}{rgb}{0.30,0.31,0.32} 
\definecolor{MyDarkBlue}{rgb}{0.,0.08,0.5} 
\definecolor{MyLightBlue}{rgb}{0.2,0.2,1.0} 
\definecolor{MyDarkRed}{rgb}{0.5,0.04,0} 
\definecolor{MyDarkGreen}{rgb}{0.0,0.4,0.08}

\definecolor{ChapterColor}{rgb}{0.16,0.32,0.75}
\definecolor{SectionColor}{rgb}{0.16,0.32,0.75}
\definecolor{SubSectionColor}{rgb}{0.0,0.53,0.74} 
\definecolor{SubSubSectionColor}{rgb}{0.0,0.58,0.65} 
%------------------------------------------------------------------------------------
% styling part pages to fit abstract 
    % https://tex.stackexchange.com/questions/30432/styling-the-part-page 
\titleclass{\part}{top}
\titleformat{\part}[display]
    {\centering\normalfont\Huge\bfseries}
    {\partname \space \thepart}
    {0pt}
    {\titlerule[1pt]\vspace{1pc}\huge}
\titlespacing*{\part}{0pt}{0pt}{20pt}

\setcounter{tocdepth}{3}
\setcounter{secnumdepth}{3}
%------------------------------------------------------------------------------------
\newcommand{\myboldverb}[1]{{\color{MyDarkBlue}\bfseries{#1}}} % per enfasi in grassetto 
\newcommand{\myverb}[1]{{\color{MyDarkBlue}\texttt{#1}}} % in evidenza
\newcommand{\myurl}[1]{{\color{MyDarkGreen}\url{#1}}} % per url
% \newcommand{\refthesischapter}[1]{{\color{ChapterColor}\textbf{Capítulo \ref{#1}}}}
% \newcommand{\refthesissection}[1]{{\color{SectionColor}\textbf{Sección \ref{#1}}}}
% \newcommand{\refthesissubsection}[1]{{\color{SubSectionColor}\textbf{Sección \ref{#1}}}}
% \newcommand{\refthesissubsubsection}[1]{{\color{SubSubSectionColor}\textbf{Sección \ref{#1}}}}
% \newcommand{\reffigure}[1]{{\color{MyDarkGreen}\textbf{Figura \ref{#1}}}}
% \newcommand{\refcode}[1]{{\color{MyDarkGreen}\textbf{Figura \ref{#1}}}}
% \newcommand{\reftable}[1]{{\color{MyDarkGreen}\textbf{Tabla \ref{#1}}}}
% \newcommand{\refequation}[1]{{\color{MyDarkGreen}\textbf{Ecuación \ref{#1}}}}
% \newcommand{\refequations}[1]{{\color{MyDarkGreen}\textbf{Ecuaciones \ref{#1}}}}

\newcommand{\refthesischapter}[1]{{\textbf{Capítulo \ref{#1}}}}
\newcommand{\refthesissection}[1]{{\textbf{Sección \ref{#1}}}}
\newcommand{\refthesissubsection}[1]{{\textbf{Sección \ref{#1}}}}
\newcommand{\refthesissubsubsection}[1]{{\textbf{Sección \ref{#1}}}}
\newcommand{\reffigure}[1]{{\textbf{Figura \ref{#1}}}}
\newcommand{\refcode}[1]{{\textbf{Figura \ref{#1}}}}
\newcommand{\reftable}[1]{{\textbf{Tabla \ref{#1}}}}
\newcommand{\refequation}[1]{{\textbf{Ecuación \ref{#1}}}}
\newcommand{\refequations}[1]{{\textbf{Ecuaciones \ref{#1}}}}

\newcommand{\textapproxtilde}{\raisebox{-0.55ex}{\textasciitilde}}

% https://tex.stackexchange.com/questions/148634/no-caption-or-title-but-entry-in-list-of-figures#148637
\DeclareCaptionLabelFormat{blank}{}
\newcommand{\blankcaption}{\captionsetup{textformat=empty,labelformat=blank} \caption{} \vspace{-1.7em}}
%------------------------------------------------------------------------------------
% \geometry{
% 	paper=a4paper, % Change to letterpaper for US letter
% 	inner=2.5cm, % Inner margin
% 	outer=3cm, % Outer margin
% 	bindingoffset=0.5cm, % Binding offset
% 	top=2.5cm, % Top margin
% 	bottom=3cm, % Bottom margin
% 	%showframe, % Uncomment to show how the type block is set on the page
% }
%---------------------------------------------------------------------------------------
% Equation float for captions: https://stackoverflow.com/questions/149479/adding-a-caption-to-an-equation-in-latex#149677
\usepackage{float}
\usepackage{aliascnt}
\newaliascnt{eqfloat}{equation}
\newfloat{eqfloat}{h}{eqflts}
\floatname{eqfloat}{Ecuación}

\newcommand*{\ORGeqfloat}{}
\let\ORGeqfloat\eqfloat
\def\eqfloat{%
  \let\ORIGINALcaption\caption
  \def\caption{%
    \addtocounter{equation}{-1}%
    \ORIGINALcaption
  }%
  \ORGeqfloat
}

%------------------------------------------------------------------------------------
%------------------------------ HERE WE GO ------------------------------------------
%------------------------------------------------------------------------------------


\begin{document}

\def\titulo{Búsqueda y Estudio Fotométrico de Sistemas Binarios Eclipsantes}
\def\autor{Ramón Caballero Villegas}
\def\matricula{2125383}
\def\grado{Maestría en Astrofísica Planetaria y Tecnologías Afines}
% En caso de que el grado tenga orientación o especialidad llenar el siguiente
% campo dejando un ESPACIO INICIAL, en caso contrario, dejar vacío
\def\orientacion{}
% Coloca el mes con mayúscula inicial
\def\fecha{Octubre 2024}

\def\asesor{Dr. Andrés Alberto Avilés Alvarado}
\def\revisorA{Nombre del revisor A}
\def\revisorB{Nombre del revisor B}
% En el caso de que tu tesis sea de doctorado activa la variable cambiándola a \doctoradotrue
% y define tus otros dos revisores
\newif\ifdoctorado\doctoradofalse
\def\revisorC{Nombre del revisor C}
\def\revisorD{Nombre del revisor D}
% El visto bueno siempre va
\def\vobo{Dr. nombre del subdirector}

%----------------------------- Titulo UANL

%%%%%%%%%%%%%%%%%%%%%%%%
% Primer portada: UANL %
%%%%%%%%%%%%%%%%%%%%%%%%
\thispagestyle{empty}

\begin{scshape}
\begin{center}
	{\Large \uanl} \\[5mm]
	{\large \fcfm} \\[5mm]
	{\large \pifi}
	\vskip15mm
	\includegraphics[height=55mm]{Figuras/uanl.png}
	\vskip12mm
	\begin{tabular}{p{11cm}}
		\centering
		{\large \titulo}
	\end{tabular}
	\vskip7mm
	{por}\\[7mm]
	{\large \autor}\\[7mm]
        % Si en tu posgrado la tesis no es opcional (como sí lo es en licenciatura), no modifiques esta línea:
	{como requisito parcial para obtener el grado de}\\[3mm]
	\MakeUppercase{\grado}\\
	\orientacion
	\vfill
	\fecha
\end{center}
\end{scshape}

%%%%%%%%%%%%%%%%%%%%%%%%%
% Segunda portada: FIME %
%%%%%%%%%%%%%%%%%%%%%%%%%
\newpage
\thispagestyle{empty}

\begin{scshape}
\begin{center}
	{\Large\uanl} \\[5mm]
	{\large\fcfm} \\[5mm]
	{\large\pifi}
	\vskip16mm
	\includegraphics[height=55mm]{Figuras/fcfm.png}
	\vskip16mm
	\begin{tabular}{p{11cm}}
		\centering
		{\large \titulo}
	\end{tabular}
	\vskip7mm
	{por}\\[7mm]
	{\large \autor}\\[7mm]
        % Si en tu posgrado la tesis no es opcional (como sí lo es en licenciatura), no modifiques esta línea:
	{como requisito parcial para obtener el grado de}\\[3mm]
	\MakeUppercase{\grado}\\
	\orientacion
	\vfill
	\fecha
\end{center}
\end{scshape}

%%%%%%%%%%%%%%%%%%%%%%%%%%%%%
% Carta del comité de tesis %
%%%%%%%%%%%%%%%%%%%%%%%%%%%%%
\newpage
\thispagestyle{empty}
\enlargethispage{5mm}

{\renewcommand{\baselinestretch}{1.1}\selectfont
\begin{center}\vspace*{-25mm}\hspace*{-10mm}
\begin{minipage}{170.5mm}
\hspace{-1.5mm}\includegraphics[height=20mm]{Figuras/uanl.png}\hfill\raise0mm\hbox{\includegraphics[height=20mm]{Figuras/fcfm.png}}
\hrule\vspace{0.5mm}
\scalebox{.5}{\MakeUppercase{\uanl}}\hfill\scalebox{.5}{\MakeUppercase{\fcfm}}\medskip
\end{minipage}
\vskip4mm{\sc\large\uanl\\\fcfm\\[3pt]\pifi}\vskip6mm
\end{center}

Los miembros del Comité de Tesis recomendamos que la Tesis <<\titulo>>, realizada por el alumno \autor, con número de matrícula \matricula, sea aceptada para su defensa como requisito parcial para obtener el grado de \grado\orientacion.
\ifdoctorado\vskip10mm\else\vskip8mm\fi

\begin{center}
El Comité de Tesis\\
\ifdoctorado\vskip15mm\else\vskip25mm\fi

\ifdoctorado{%%%
\begin{tabular}{p{37mm}p{21mm}p{12mm}p{21mm}p{37mm}}
	\cline{2-4}
	& \multicolumn{3}{c}{\asesor} & \\
	& \multicolumn{3}{c}{Asesor}  & \\[15mm]
	\cline{1-2} \cline{4-5}
	\multicolumn{2}{c}{\revisorA} & & \multicolumn{2}{c}{\revisorB} \\
	\multicolumn{2}{c}{Revisor}   & & \multicolumn{2}{c}{Revisor}   \\[17mm]
	\cline{1-2} \cline{4-5}
	\multicolumn{2}{c}{\revisorC} & & \multicolumn{2}{c}{\revisorD} \\
	\multicolumn{2}{c}{Revisor}   & & \multicolumn{2}{c}{Revisor}   \\[2mm]
	& \multicolumn{3}{c}{Vo. Bo.} & \\[14mm]
	\cline{2-4}
	& \multicolumn{3}{c}{\vobo}   & \\
	& \multicolumn{3}{c}{Subdirector de Estudios de Posgrado}   & \\ &&&&
\end{tabular}
}\else{%%%
\begin{tabular}{p{37mm}p{21mm}p{12mm}p{21mm}p{37mm}}
	\cline{2-4}
	& \multicolumn{3}{c}{\asesor} & \\
	& \multicolumn{3}{c}{Asesor}  & \\[19mm]
	\cline{1-2} \cline{4-5}
	\multicolumn{2}{c}{\revisorA} & & \multicolumn{2}{c}{\revisorB} \\
	\multicolumn{2}{c}{Revisor}   & & \multicolumn{2}{c}{Revisor}   \\[2mm]
	& \multicolumn{3}{c}{Vo. Bo.} & \\[17mm]
	\cline{2-4}
	& \multicolumn{3}{c}{\vobo}   & \\
	& \multicolumn{3}{c}{Subdirector de Estudios de Posgrado}   & \\ &&&&
\end{tabular}
}\fi%%%

\vfill

\snnl, \MakeLowercase{\fecha}

\end{center}}


% ------------------------------------DEDICATION
% \restoregeometry
% \vspace*{5cm}
% \begin{flushright}
%   {\parbox{4.2cm}{\textit{Dedication...}}}
% \end{flushright}

% \thispagestyle{empty}
% ------------------------------------
%\newpage
%-------------------------------------ABSTRACT
\thispagestyle{empty}
% \vspace*{5cm}
\renewcommand{\abstractname}{\color{MyDarkBlue}{Resumen}}
% \begin{abstract}
    % TODO: new abstract about contact binaries

    % Las \textit{variables cataclísmicas} son un tipo de sistemas binarios con un
    % comportamiento particular. Estas están compuestas comúnmente de una estrella
    % \textit{enana blanca} y una de \textit{secuencia principal}, donde material
    % fluye de la secuencia principal hacia la enana blanca, por lo cual se les
    % llama sistemas en contacto. De esto surge el fenómeno de las explosiones
    % \textit{nova}, en donde el material acrecido es lanzado de manera violenta
    % al medio interestelar, causando un aumento en el brillo del sistema. Este
    % abrillantamiento es una de las características más llamativas de las
    % variables cataclísmicas, siendo un comportamiento volátil y periódico cuyo
    % mecanismo no ha sido completamente explorado. 
    % \\\newline
    % \noindent A pesar de ser objetos de interés son relativamente pocas los
    % sistemas identificados como VCs en la literatura. Hasta recientemente solo
    % se han podido encontrar poco más de 1,000 VCs; esto limita posibles
    % observaciones a una parte pequeña de la población en la Galaxia, lo cual
    % causa un desacuerdo en los mecanismos en juego en la evolución de estos
    % mismos. Cada nuevo sistema observado provee datos útiles que ayudan a
    % desarrollar un modelo general no solo para las VCs, si no que también para
    % la estructura y evolución de estrellas solas como nuestro Sol. 
    % \\\newline
    % \noindent Esta tesis tiene de objetivo encontrar candidatas a VCs dentro del
    % catálogo de \gaiaNoSpace, implementando una búsqueda fotométrica basado en
    % filtros de colores de las bandas del \textit{Sloan Digital Sky Survey}, con
    % énfasis en encontrar sistemas previamente no observados. De los sistemas
    % candidatos se llevo a cabo un estudio fotométrico desde el Observatorio
    % Astronómico Universitario de la Universidad Autónoma de Nuevo León, ubicado
    % en el Municipio de Iturbide Nuevo León. Como resultado final de este trabajo
    % se presenta las curvas de luz de cada objeto, obteniendo de estas varios
    % parámetros efemérides de los sistemas.
% \end{abstract}
%-------------------------------------INDEX
\setcounter{page}{1}
\tableofcontents
\setcounter{secnumdepth}{4}
\setcounter{tocdepth}{4}
%\vspace*{2cm}
%\setcounter{page}{1}
\pagenumbering{arabic}

%%%%%%%%%%%%%%%%%%%%%%%%%%%%%%%%%%%%% Chapters
% \chapter{Introducción}
\part{Introducción}

% TODO: make abstract

\section{Estrellas}

Las \textbf{estrellas} son de los objetos más fundamentales e importantes en el
estudio de los astros. \citetbookchapter{anIntroStellarAstro}{1} define una
estrella como \quotes{un objeto celeste en el cual existe, o alguna vez existió
(en el caso de una estrella muerta) fusión de hidrógeno sostenido en su núcleo.}
El núcleo de cada estrella activa alcanza temperaturas en el orden de decenas de
millones Kelvin, lo cual permite forjar nuevos elementos más pesados mediante el
proceso de fusión nuclear. El mecanismo principal para la generación de energía
es la fusión de hidrógeno a helio: $\ce{4 ^1H} \rightarrow \ce{^4He} + E$ .
Dentro de todas las estrellas existe un balance de fuerzas que mantiene la forma
esférica de la estrella; esto se le conoce como el \textit{equilibrio
hidrostático}, en el cual la presión interna de la estrella tiene una
contra-fuerza equivalente del peso de su mismo gas. Este balance es modelado a
escalas menores a las de las propiedades macroscópicas de la estrella
\citetbookchapter{anIntroTheoryStellarStructureEvolution}{2}, en la cual la
estrella está en un estado de \textit{equilibrio termodinámico local}. 

Una estrella es caracterizada por sus propiedades físicas, como su masa,
composición química, radio, y temperatura. A continuación se planta una base
para describir la estructura de una estrella al igual que su evolución,
empezando con la formación de una \textit{protoestrella}, la cual a lo largo del
tiempo se transforma en una estrella.

\subsection{Formación}

% TODO: add citations from "Formación estelar" presentation I did in MHD course
% Agrega paginas web

El \textbf{medio interestelar} (ISM por sus siglas en inglés) es la región de
espacio que existe entre el Sol y el resto de las estrellas dentro de nuestra
Galaxia. Está compuesto de todo el polvo, gas, y partículas libres como los
rayos cósmicos que atraviesan el espacio. La distribución de este material no es
uniforme a cortas escalas de distancias astronómicas, por lo cual se observan
acumulaciones de gas y polvo que se les conoce como \textbf{nubes moleculares}.
Estas nebulosas se encuentran a temperaturas extremadamente frías, hasta los
10-20 K en su estado de equilibrio. Sin embargo, la nube puede llegar a ser
perturbada de este estado, lo cual causa una distribución de masa y un campo
gravitacional no uniforme, dando forma a cúmulos densos de material.

% TODO: probably add symbols table in thesis for M_\odot (and probably others to
% be used in rest of doc)

Dado suficiente tiempo una nube molecular va colapsando bajo su propio peso,
donde la presión en la nube causa que el material se acumule en un solo punto en
el espacio. Mientras más aumente la presión el gas se calienta; este proceso
sigue hasta llegar a un equilibrio entre la presión interna del gas y la fuerza
compresiva de gravedad, en donde la nube deja de contraerse. Dependiendo de la
temperatura a la que alcanza el núcleo determina si se convierte en una estrella
capaz de mantener una reacción continua de fusión nuclear. La masa mínima para
que una nube molecular colapse a una estrella es de $0.08 \ \mathrm{M}_{\odot}$
\citetbookchapter{anIntroStellarAstro}{2}. De no cumplir con esta condición, el
cúmulo de gas y polvo no logrará mantener la cadena de reacciones termonucleares
en su núcleo, resultando en una \textit{enana café}, un objeto inerte resultado
del proceso fallido de formación estelar.

Una vez que el material de la nube se estabilice\textemdash una vez que el
objeto empiece a producir energía de orden de magnitud estelar\textemdash nace
una \textbf{protoestrella}. Estos objetos extensos (aproximadamente $500 \
\mathrm{R}_{\odot}$ para una protoestrella de $1 \ \mathrm{M}_{\odot}$
\citetbookchapter{astronomyPhysicalPerspective}{15}) se mantienen a una baja
temperatura durante esta etapa de su vida, alrededor de 2500 K en su superficie.
Sin embargo, debido a la alta densidad, la nube no es completamente
transparente, por lo cual solo la superficie es capaz de liberar energía por
medio de radiación infrarroja; la energía producida en su interior primero debe
de viajar a la superficie por medio de convección. Conforme va evolucionando la
nube se sigue comprimiendo hasta llegar a un punto en el cual puede mantener
reacciones termonucleares continuas en su núcleo, llegando al punto de ser una
estrella en \textbf{secuencia principal}. La \reffigure{protostarEvolutionFig}
muestra la evolución de una protoestrella a una estrella secuencia principal con
respecto a su temperatura y luminosidad.

\begin{figure}[!ht]
	\centering
	\includegraphics[scale=0.3]{Introduccion/Figures/EvolucionZAMSFormacionEstelar_Kutner.png}
	\caption{Diagrama de la evolución de una protoestrella hasta llegar a
	\textit{ZAMS} (\textit{Zero Age Main Sequence}), la edad a la que se
	convierte en una estrella de secuencia principal. Se muestran varios caminos
	evolutivos de una protoestrella en función de su masa, el cual es el
	parámetro de mayor importancia en su evolución y estado final. Figura
	obtenida de \citetbookchapter{astronomyPhysicalPerspective}{15}.}
	\label{protostarEvolutionFig}
\end{figure}

\subsection{Balance de Fuerzas}

Una vez que una estrella llegue su etapa de secuencia principal el sistema se
vuelve estable a largas escalas temporales. En este estado la estrella deja de
comprimirse, ya que el colapso gravitacional es contrarrestado por la presión
del gas caliente en su interior y la presión ejercida por la radiación producida
en su núcleo. A continuación se definen dos conceptos importantes en la física
estelar: el \textbf{equilibrio hidrostático} y el \textbf{equilibrio
termodinámico local}.

\subsubsection{Equilibrio Hidrostático}

Podemos simplificar el modelo de una estrella al introducir la simetría
esférica, en el cual las propiedades del gas que forma la estrella dependen
unicamente de la distancia del núcleo. Para esto definimos la presión $P(r)$,
densidad $\rho (r)$, temperatura $T(r)$, y aceleración gravitacional $g(r)$ como
funciones con respecto a la distancia $r$ del centro. En base a estos
fundamentos se puede deducir la estratificación de una estrella, formando varias
capas de material con propiedades uniformes a lo largo de cada estrato. Dado que
cada elemento está sujeto a un equilibrio de fuerzas descrita por la ecuación
$P(r) \mathrm{d}A - [P(r) + \mathrm{d}P] \mathrm{d}A - [\rho(r) \mathrm{d}A
\mathrm{d}r] g(r) = 0$, la cual se ve en la
\reffigure{figuraEquilibrioHidrostatico}. Al simplificar esta ecuación e
introduciendo un término para la presión radiativa\textemdash debido a la
transferencia de momento de los fotones generados en el núcleo estelar al
material circundante\textemdash se obtiene la
\refequation{ecuacionEquilibrioHistrostatico}. 

\begin{eqfloat}[!ht]
	\centering
	\begin{equation}
		\frac{\textrm{d}P(r)}{\textrm{d}r} = -\rho(r) [g(r) - g_{rad}(r)]
	\end{equation}
	\caption{Ecuación del equilibrio hidrostático para una estrella, tomando en
	cuenta los efectos de la presión radiativa saliente $g_{rad}(r)$.
	\citetbookchapter{anIntroStellarAstro}{2}}
	\label{ecuacionEquilibrioHistrostatico}
\end{eqfloat}

\begin{figure}[!ht]
	\centering
	\includegraphics[scale=0.5]{Introduccion/Figures/EquilibrioHidrostatico_LeBlanc.png}
	\caption{Diagrama de fuerzas actuando sobre cada elemento superficial
	$\mathrm{d}A$ de material de una estrella. Los efectos debido a elementos
	adyacentes son nulos, debido a que la presión en cada lados es constante,
	resultando en un gradiente de presión radial. Figura obtenida de
	\citetbookchapter{anIntroStellarAstro}{2}.}
	\label{figuraEquilibrioHidrostatico}
\end{figure}

\subsubsection{Equilibrio Termodinámico Local}

\subsection{Evolución}

En el transcurso del tiempo cada estrella será sujeta a ciertos cambios en su
estructura. Esto se debe a que podemos considerar a una estrella cómo un objeto
aislado en el espacio, lo cual significa que no tendrá algún ingreso de material
significativo para reemplazar el combustible \quotes{quemado} en las reacciones
termonucleares. A lo largo del tiempo la composición física y química de la
estrella deberán cambiar para mantener el equilibrio termodinámico.

El combustible primario de una estrella viene siendo el hidrógeno atómico, el
cual se fusiona con otros átomos (protones individuales) libres, resultando en
la producción de grandes cantidades de energía en forma de radiación y moléculas
de helio como producto. El helio no es inmediatamente util para la estrella cómo
combustible, ya que requiere temperaturas más altas de las que se encuentran
durante esta fase evolutiva. Todas las estrellas conocidas pasan por esta etapa
de evolución estelar; mientras que una estrella dependa principalmente del
hidrógeno para brillar se dice que está en su etapa de \textit{secuencia
principal}. 

\begin{figure}[!ht]
	\centering
	\includegraphics[scale=0.6]{Introduccion/Figures/Figura Evolucion_MS_Astronomy_Physical_Perspective.png}
	\caption{Evolución de estrellas de la secuencia principal basado en su masa
	inicial en el diagrama HR. La línea punteada representa la posición de la
	estrella en el primer momento que se integra a la secuencia principal. Al
	consumir el hidrógeno en su núcleo por las reacciones nucleares que ocurren
	en esta misma región se comienza a desatar el equilibrio delicado que
	mantiene la forma de la estrella. Esta deformación provoca una oscilación en
	su tamaño, causado por las fluctuaciones del balance entre la presión
	radiativa generada por las reacciones nucleares en el núcleo contra la
	presión gravitacional. Diagrama obtenido de
	\citet{astronomyPhysicalPerspective_stellarOldAgeChapter}}
	\label{evolucionMSEstrella}
\end{figure}
% \section{Sistemas Binarios}
\chapter{Sistemas Binarios}

La gran mayoría de sistemas estelares dentro de nuestra Galaxia no son aquellos
solitarios como nuestro propio sistema solar, si no que son compuestas de dos o
más estrellas ubicadas en corta aproximación de una a otra, a ordenes de
unidades astronómicas (AU por sus siglas en inglés). Estos sistemas multiples se
pueden clasificar con mayor precisión para aquellos compuestos de solo dos
estrellas, denominados como \textit{sistemas binarios}. Dentro de un sistema
binario la corta separación orbital entre ambas estrellas da como consecuencia a
fenómenos que surgen mediante la interacción entre las componentes, tanto como
la interacción gravitacional debido a sus masas, como a la física interesante
que ocurre en el caso de interacciones de material entre una estrella a otra. 

% TODO: incluir sección en cite?
Los sistemas binarios estelares ofrecen un laboratorio celeste de gran
importancia, ya que debido a la interacción a cortas escalas espaciales nos
brindan información acerca de las estrellas que sería imposible obtener de otra
manera. \citetbookchapter{phoebeScientificReference}{2.1} menciona varios
parámetros derivados de observaciones de estos sistemas, como las masas de cada
una de las componentes estelares por medio de su interacción gravitacional, la
calibración y estudio de la evolución estelar y su dependencia de la masa y
luminosidad de la estrella; dadas observaciones de sistemas desconectados, en el
cual ambas componentes son de la misma edad pero con diferentes propiedades que
influyen su camino evolutivo. La clasificación de estos sistemas se basa tanto
en propiedades observacionales\textemdash las cuales dependen tanto de las
propiedades geométricas del sistema como de nuestra capacidad de observación, en
cuanto a la capacidad de la instrumentación disponible\textemdash como en las
propiedades físicas del sistema, incluyendo la proximidad de las componentes
como sus propiedades lumínica.

\section{Geometría del Sistema - Modelo de Roche}

Es importante entender la forma geométrica de un sistema binario para llegar a
una descripción adecuada de ellos. Esto incluye los parámetros orbitales de las
estrellas tanto como la forma misma de ambas componentes, ya que en ciertos
casos tratar las estrellas como esferas rígidas da como resultado un modelo
incorrecto. A continuación se introduce las bases de las cuales se parte para
llegar a una representación de un sistema modelo, llegando a describir el
\textbf{modelo de Roche}.

Se define un sistema de coordenadas cartesiano tridimensional considerando un
marco de referencia no inercial, el cual está rotando con la misma velocidad que
las componentes del sistema binario orbitan una a otra. Esta es una
representación típica para el \textit{problema de tres cuerpos}, en el cual
tenemos a dos objetos masivos cuya influencia gravitacional se extiende al
espacio representado por el sistema de coordenadas. En la
\reffigure{figuraTresCuerpos} se puede ver este esquema, donde $m_1$ y $m_2$
representan ambas componentes estelares, posicionadas de tal manera que la
distancia entre las estrellas solo tenga una componente en una dirección
cartesiana.

\begin{figure}
	\centering
	\includegraphics[scale=0.4]{Introduccion/Figures/Figura Tres Cuerpos_Intro Evolution Single Binary Stars.png}
	\caption{Configuración del problema de tres cuerpos dados dos objetos de
	alta masa $m_1$ y $m_2$ (las cuales representan cada componente estelar del
	sistema binario), y una partícula de masa despreciable de prueba ubicada a
	una distancia $r_1$ y $r_2$ de las estrellas respectivamente. El centro de
	masa del sistema está ubicado en el origen del sistema de coordenadas;
	\citetbookchapter{phoebeScientificReference}{3.1} posiciona la componente
	$m_1$ en el origen, con tal de simplificar las expresiones del modelo.
	Figura obtenida de
	\citetbookchapter{benacquista_introduction_to_evolution_single_binary_stars_2013}{13.1}.}
	\label{figuraTresCuerpos}
\end{figure}

La \reffigure{figuraTresCuerpos} define el sistema de coordenadas rotando junto
al sistema binario. La velocidad angular de Kepler está dada por:

\begin{eqfloat}
	\centering
	\begin{equation}
		\omega = \frac{2 \pi}{P_{orb}} = \sqrt{\frac{G (m_1 + m_2)}{a^3}}
	\end{equation}
	\blankcaption
	\label{ecuacionVelocidadAngular}
\end{eqfloat}

Donde $G = 6.673 \times 10^{-11} \ \mathrm{m}^3 \mathrm{kg}^{-1}
\mathrm{s}^{-2}$ es la constante de gravitación universal, $P_{orb}$ es el
periodo orbital de las estrellas (el cual es igual para ambas estrellas de
acuerdo con la segunda ley de Kepler), y $a = a_1 + a_2$ es el semieje mayor.
Usando la \refequation{ecuacionVelocidadAngular} podemos definir el Lagrangiano
para una partícula de prueba con masa $m$, el cual está a una distancia
$\mathbf{r}_1$ y $\mathbf{r}_2$ de $m_1$ y $m_2$ respectivamente, y a una
distancia $\mathbf{r}$ del centro de masa del sistema:

\begin{eqfloat}
	\centering
	\begin{equation}
		\Lagr = \eqnmark[MyDarkRed]{node1}{\frac{1}{2} m (\dot{x}^2 + \dot{y}^2) + \dot{z}^2}  +
				\eqnmark[MyDarkGreen]{node2}{\frac{1}{2} m \omega^2 (x^2 + y^2)} +
				\eqnmark[MyDarkBlue]{node3}{\frac{G m_1 m}{r_1} + \frac{G m_2 m}{r_2}}		
	\end{equation}
	\blankcaption
	\label{ecuacionLagrangianoTresCuerpos}
\end{eqfloat}

\section{Clasificaciones Observacionales}

Dependiendo del método de detección y las propiedades aparentes del sistema se
puede clasificar un sistema binario de estrellas. Estas clasificaciones son
independiente de sus propiedades físicas, como la clase espectral de cada
estrella o sus masas individuales. Al determinar su clasificación observacional
se puede delimitar las técnicas observacionales que son viables para recabar
datos del sistema; un sistema astrométrico sería indistinguible de uno
espectroscópico si uno intenta identificar las componentes individuales a simple
vista, o con un telescopio demasiado débil para el trabajo.

Las \textbf{binarias visuales} son aquellos cuya separación orbital aparente es
suficientemente grande para distinguir las dos estrellas individuales en la
bóveda celeste. A pesar de que se puede trazar la órbita de la secundaria con
varios años de observaciones, se requiere de cálculos adicionales para
determinar la órbita exacta de las componentes. Esto se debe a la inclinación
del sistema con respecto al eje de observación hacia la Tierra; solo es posible
observar \quotes{una proyección del elipse orbital relativo en el plano del
cielo,} aunque esto se puede superar usando el hecho de que la estrella primaria
aparentemente inmóvil debe de estar presente \quotes{en un punto focal de la
órbita relativa.} \citetbookchapter{fundamentalAstronomy}{10}

Las \textbf{binarias espectroscópicas} presentan variaciones periódicas en sus
espectros, en donde las líneas espectrales detectadas \quotes{oscilan
periodicamente alrededor de la longitud de onda promedio}
\citetbookchapter{astronomyPhysicalPerspective}{5}. Esto se observa
debido al \textit{desplazamiento de Doppler}, lo cual causa que la frecuencia de
un fotón se recorra hacia frecuencias más pequeñas (azules) o más grandes
(rojas) dependiendo de su velocidad radial con respecto al observador, si se va
acercando o alejando, respectivamente. Estas también pueden identificadas al
observar dos distintos grupos de líneas espectrales, el cual es resultado de la
contribución de ambas estrellas.

Las \textbf{binarias astrométricas}, al igual que las espectroscópicas, solo
muestran una componente visible al ser observada, al contrario de las binarias
visibles. Sin embargo, una binaria astrométrica difiera de las otras dos
categorías definidas en cuestión de su movimiento observado en la bóveda
celeste. Estas muestran un movimiento errático y no-lineal, algo que no se
esperaría ver en una estrella solitaria dado su inercia según la primera ley de
Newton. Estas perturbaciones son causadas por una estrella secundaria no
aparente al observar el sistema. 

\section{Binarias Eclipsantes}

Una de las propiedades más importantes de identificar de un sistema binario es
la \textit{inclinación} de su órbita con respecto a nuestra línea de visión
desde el sitio de observación (ya sea la Tierra en caso de un observatorio
terrestre o un punto lejano dentro del sistema solar para un telescopio
espacial). En dado caso que un sistema tenga una inclinación suficientemente
alta se pueden observar eclipses dentro del sistema, en lo que una componente
obscurece a su compañera de nuestra línea de visión. Estos eclipses se
manifiestan como disminuciones en el flujo total del sistema, brindando
información que no se obtendría sin esta perspectiva visual. 

\section{Binarias en Contacto}
% \section{Variables Cataclísmicas}

\input{Introduccion/Subsecciones/CV/Clasificacion.tex}
\subsection{Evolución}

En el transcurso del tiempo una estrella va a estar sujeta a ciertos cambios en
su estructura característica. Esto se debe a que una estrella la podemos
considerar cómo un objeto aislado en el espacio, lo cual significa que no tendrá
algún ingreso de material significativo para reemplazar el combustible
\quotes{quemado} en las reacciones termonucleares. Dado esto, la evolución de
una estrella se considera en base a su estado termodinámico, y los procesos que
ocurren mientras que la estrella intenta regresar a un estado de equilibrio. 
\subsection{Disco de Acreción}

\comment{
Nuestra Galaxia es el hogar de varios sistemas solares, los cuales muestran una
gran variedad de propiedades físicas. Existen sistemas tal como nuestro sistema
solar, que consisten de una estrella de secuencia principal rodeada de planetas
y restos de material de su etapa de formación. Otros sistemas están compuestos
de dos estrellas o más, donde todas las estrellas del sistema orbitan un punto
en común denominado el \textit{centro de masas}. Esta tesis se enfoca en un tipo
de sistema en especifico denominado como \textbf{variables cataclísmicas}:
sistemas de contacto que consisten de una estrella de tipo
\hyperref[intro:sec:EnanaBlanca]{enana blanca} y una
\hyperref[intro:sec:EnanaRoja]{enana roja}. 
}

\comment{
Este tipo de sistemas muestran comportamiento variable a lo largo del tiempo, en
donde varía su luminosidad, espectro emitido, y otras propiedades del sistema.
No todas las variables cataclísmicas demuestran un comportamiento uniforme; sus
periodos de intensa actividad y el subsecuente periodo de atenuación pueden
variar en su duración y su magnitud. El mecanismo que genera esta variación se
les conoce como \textit{novae}; estos estallidos ocurren debido a la interacción
de las estrellas dentro del sistema.
}

\comment{
\subsection*{Composición de las VCs}
La composición de las estrellas puede variar de un sistema a otro, pero en
general adhieren a los siguientes parámetros: una estrella enana blanca -
conocida como la estrella principal - y una estrella de secuencia principal
menos densa - la estrella secundaria. Esta estrella secundaria es más roja que
la principal; su espectro de emisión tiende a la región roja, llegando hasta el
infra-rojo en ciertos casos. Tomando en cuenta el sistema completo, se puede
observar en la distribución de variables cataclísmicas en el diagrama
Hertzsprung-Russell que la mayoría reside entre las regiones de enanas blancas y
la secuencia principal dependiendo de la contribución relativa de sus estrellas
componentes. \citet{disentanglingGaiaHR} No siempre ocurre algún tipo de
interacción entre las estrellas componentes dentro del sistema; esto va a
depender de los volúmenes de las estrellas y la distancia entre ellas, ya que la
manera principal en que interactúan entre si es en base a su atracción
gravitacional. Entre más cerca de una a otra las fuerzas gravitacionales
empiezan a distorsionar la secundaria hasta tener una forma más plana y menos
uniforme. Este volumen, llamado el lóbulo de Roche, es la región de espacio en
donde la estrella puede mantener a su material atrapada por su gravedad. En un
sistema binario cada estrella tiene un lóbulo de Roche definido, los cuales
están en contacto uno con el otro. Un ejemplo simple se puede ver en la figura
\ref{acrecionSmithReview}, donde la estrella secundaria ha llenado su volumen y
su material empieza a caer hacia la enana blanca debido a su atracción
gravitacional. \\\newline
Al cruzar el lóbulo de Roche de la estrella secundaria, la materia perdida no
puede caer directo a la superficie de la enana blanca a ser absorbida
inmediatamente. Esto se debe al momento angular de las estrellas, tanto por su
rotación como su órbita alrededor de su estrella compañera. Por lo tanto, al
cruzar el punto de contacto entre los lóbulos (este conocido como el punto
Lagrangiano L1) el material tiene una componente de velocidad tangente a la
dirección a la enana blanca. A lo largo de que el material se empieza a acumular
toma una forma dependiendo de las propiedades magnéticas de la enana blanca. En
un sistema donde no ejerce un efecto significativo un campo magnético, el
material forma un \textit{disco de acreción} alrededor de la enana blanca. Este
disco se concentra en el ecuador de esta misma estrella, el cual empieza a
emitir radiación en forma de líneas de emisión. Al contrario en un sistema
magnético (llamados \textit{variables cataclísmicas magnéticas} o
\textit{polares}) donde el campo magnético de la enana blanca es significativo
el material no logra acumularse en un solo disco. Este material debido a ser un
plasma ionizado tiene una carga eléctrica intrínseca, la cual la permite ser
guiada por el campo magnético de la enana blanca a su destino final al polo
magnético de esta misma estrella. \\\newline
\begin{figure}
	\centering
	\includegraphics[scale=0.4]{Introduccion/Figures/Figura Acrecion_SmithReview.png}
	\caption{Diagrama de una estrella enana blanca absorbiendo material de su
		compañera estrella de secuencia principal por medio de la interacción de sus
		lóbulos de Roche.} \citet{smithReview}
	\label{acrecionSmithReview}
\end{figure}

Todas las variables cataclísmicas han obtenido su nombre debido a sus cambios
rápidos en sus propiedades durante periodos de observación. El mecanismo
principal que causa estas variaciones se debe a los estallidos periódicos,
conocidos como \textit{novae}. Al igual como no todos los sistemas muestran las
mismas propiedades en sus componentes existen diferentes tipos de estallidos:
\textit{novae clásicos}, \textit{novae enanos}, y \textit{nova-like}.
\cite*{smithReview} \\\newline
Los novae clásicos son aquellos estallidos observados en sistemas binarios que
causa que su luminosidad aumente por varias magnitudes en una escala de tiempo
relativamente corta (puede ser en la escala de días hasta meses).
\citet{smithReview} Aparte de la radiación generada debido a reacciones
termonucleares, los novae expulsan un entorno de gas caliente a grandes
velocidades, llegando a miles de kilómetros por segundo. Pocos sistemas binarios
muestran nova recurrentes a lo largo del tiempo en intervalos de decenas o
cientos de años; la mayoría ha mostrado un solo estallido en su tiempo de
observación. En sistemas no magnéticos, el disco de acreción sirve como el motor
principal de estos estallidos. La mayoría del material en el disco es hidrógeno;
este crea una capa en la superficie de la enana blanca. A lo largo de lo que se
acumula este material aumente su grosor, aumentando la temperatura de la capa.
Este aumento de temperatura causa que la presión en esta capa alcance un punto
en donde empiezan a ocurrir reacciones termonucleares de parte del hidrógeno
presente en esta capa.
}
% \subsection*{Enana Blanca}

Una estrella nace de una nube molecular interestelar, una región de material
ubicada en el espacio entre estrellas. Dependiendo de la masa inicial de el
conjunto inicial de material es lo que determina las fases que la estrella pasa
al envejecer. El camino que tomaría una estrella de secuencia principal durante
el fin de su vida se puede ver en la figura \ref{evolucionMSEstrella}, donde
está marcado los distintos caminos que una estrella toma en el \textbf{diagrama
Hertzsprung-Russell (HR)}. Este diagrama relaciona la temperatura efectiva de la
estrella con su luminosidad, dada en términos de luminosidad solar.

\begin{figure}[!ht]
	\centering
	\includegraphics[scale=0.5]{Introduccion/Figures/Figura Evolucion_MS_Astronomy_Physical_Perspective.png}
	\caption{Evolución de estrellas de la secuencia principal basado en su masa
	inicial en el diagrama HR. La línea punteada representa la posición de la
	estrella en el primer momento que se integra a la secuencia principal. Al
	consumir el hidrógeno en su núcleo por las reacciones nucleares que ocurren
	en esta misma región se comienza a desatar el equilibrio delicado que
	mantiene la forma de la estrella. Esta deformación provoca una oscilación en
	su tamaño, causado por las fluctuaciones del balance entre la presión
	radiativa generada por las reacciones nucleares en el núcleo contra la
	presión gravitacional.}
	\citet{astronomyPhysicalPerspective_stellarOldAgeChapter}
	\label{evolucionMSEstrella}
\end{figure}

Aquellas estrellas cuyas masas iniciales recae bajo 8.5-10.6 \(M_{\odot}\)
terminan su vida como una estrella \textit{enana blanca.}
\citet*{whiteDwarfsReview} El ciclo de reacciones nucleares dentro del núcleo de
una estrella solo ocurre en la presencia de cierta cantidad de hidrógeno durante
su tiempo en la secuencia principal; al acabarse esta fuente de combustible la
estrella empieza a colapsar en si misma, ya que la presión radiativa del núcleo
hacia el exterior disminuye a tal grado que la presión hacia el interior de su
propia gravedad causa el encogimiento de la estrella. Esta disminución de su
radio causa que el núcleo se caliente hasta llegar a temperaturas \(T \approx
10^{8} K\) [\citet*{astronomyPhysicalPerspective_stellarOldAgeChapter}],
empezando de nuevo reacciones nucleares, esta vez involucrando el helio en vez
del hidrógeno. Estas reacciones se conocen como el proceso \textit{triple alfa},
donde tres partículas alfa \(^{4}He\) fusionan para crear un átomo de \(^{12}C\)
y un fotón gama. Mientras que en el núcleo ocurren reacciones con elementos cada
vez más pesados, el resto de los elementos más livianos (ya sea hidrógeno en el
caso de estrellas sometidas al proceso alfa u objetos más pesados como neon u
oxígeno en el caso de núcleos más densos
\citet*{astronomyPhysicalPerspective_stellarOldAgeChapter}) siguen presentes en
las capas que rodean al núcleo. La energía generada por las reacciones nucleares
dentro del núcleo se transporta a estas capas externas por medio de la radiación
generada, la cual calienta los elementos livianos, desatando de nuevo la fusión
de elementos como hidrógeno. Estas explosiones en las capas exteriores de la
estrella causa la expansión de la estrella, llegando a la fase gigante dentro
del diagrama HR.

Dependiendo de su masa inicial, una estrella puede seguir produciendo elementos
cada vez más pesados dentro de su núcleo. Sin embargo, este combustible solo le
permite a la estrella llegar hasta cierta temperatura, a partir de cual no podrá
seguir manteniendo su tasa de fusión nuclear. Una vez que llegue a este punto
empieza a expulsar las capas exteriores hasta solo dejar el núcleo expuesto,
ahora inerte debido a la ausencia de fusión nuclear. Este resto de la estrella
progenitora es lo que se conoce como la \textit{enana blanca}, a pesar de no ser
una estrella formalmente. La composición del material dentro de este objeto es
distinto al de una estrella de secuencia principal; a pesar de tener como mínimo
una masa \textapproxtilde 0.30-0.45 \(M_{\odot}\) su radio en promedio cae
dentro del mismo orden de magnitud que el radio de la Tierra.
\citet*{whiteDwarfsReview} Esto implica una densidad inmensa, en donde solo un
\textit{gas degenerado de Fermi} puede existir en estas condiciones. Un gas
degenerado surge como consecuencia del \textbf{principio de exclusión de Pauli}:
dentro de una molécula no pueden existir más de un electrón por cada estado
cuántico. Es debido a éste fenómeno que las moléculas de una estrella enana
blanca están acumuladas en un volumen varias ordenes menor comparado con una
estrella de secuencia principal, en la cual los electrones degenerados les
permiten a las moléculas almacenar más energía térmica de lo que predicen los
modelos en un gas no degenerado. Por lo tanto la escala termodinámica de una
estrella enana blanca puede llegar a ordenes de \textapproxtilde \(10^{10}\)
años, en la cual su temperatura efectiva podría disminuir de \(100,000 K\) a
\textapproxtilde \(5,000 K\). \citet*{whiteDwarfsReview}
% \subsection{Enana Roja} \label{intro:sec:EnanaRoja}


\chapter{Muestra}

Este trabajo tiene como objetivo principal producir datos fotométricos de
estrellas marcadas como candidatas a variables cataclísmicas. Debido a la
escasez de variables cataclísmicas en la literatura---solo 1093 variables
cataclísmicas habían sido identificadas hasta 2015 en el catálogo de Ritter-Kolb
\citet*{ritterKolbImpactArticle}---existen pocos datos con los cuales corroborar
los modelos actuales de una variable cataclísmica. Al identificar nuevas
estrellas como variables cataclísmicas y generar una curva de luz de estas
mismas estaremos aportando a la muestra disponible en la literatura. Con este
fín utilizamos el catálogo de \gaia para obtener nuestras candidatas, usando el
catálogo de SDSS para seleccionar solo las fuentes no observadas previamente. 

\section{Catálogos}
\section{Gaia} \label{muestra:sec:gaia}

La misión de \gaia fue lanzada por la \textbf{Agencia Espacial Europea (ESA)} el 19 de Diciembre del 2013, con el objetivo de generar un mapa tridimensional de nuestra Galaxia, la Vía Láctea. Esto incluye calcular las propiedades astrométricas y astrofísicas de sus fuentes observadas con mayor precisión que cualquier otro catálogo publicado previamente. Para lograr esto se utiliza un satélite espacial, el cual está denominado como \gaiaNoSpace, ubicado en el punto Lagrangiano L2 con respecto al sistema Sol-Tierra. Desde este punto la nave tiene una vista sin obstrucciones que le permite observar una cantidad de estrellas enorme, con \textapproxtilde 1,000 millones de fuentes visibles con los instrumentos del satélite \gaiaNoSpace. \citet*{gaiaMission}

\subsection{Data Release 2} \label{muestra:sec:gaia:dr2}

\subsection{Sloan Digital Sky Survey}

La colección de catálogos \textbf{Sloan Digital Sky
Survey}\footnote{\url{https://www.sdss.org}} (de ahora en adelante será referido
como \textbf{SDSS}) compila varias fuentes de datos astronómicos y astrofísicos
en un sitio centralizado, con el objetivo de crear un mapa tridimensional del
Universo con una precisión no vista antes. Estos incluyen imágenes de objetos
astronómicos en varios colores, acompañados de un espectro obtenido como parte
de esta misión. Para los finales del siglo XX habían surgido avances
tecnológicos que llegarían a revolucionar la astronomía observacional. De estos,
los de mayor interés ocurrieron con los detectores de estado sólido y en la
capacidad computacional de procesamiento. Partiendo de estos empezaron a
desarrollar la infraestructura necesaria para recabar datos fotométricos y
espectroscópicos.

El instrumento principal utilizado es el telescopio de 2.5m, ubicado en el
observatorio \textit{Apache Point Observatory}, descrito a detalle en
\citet*{sdss2_5mTelescope}. Este telescopio de diseño de Ritchey-Chrétien
alimenta dos instrumentos separados; una CCD multi-banda de ancha área, y un par
de espectrógrafos alimentados por fibra óptica. Su construcción empezó en 1998,
pero no fue hasta el año 2000 que estuvo operacional.

\subsubsection{Data Release 9}

SDSS libera datos en colecciones iterativas; es decir cada Data Release (DR)
liberado contiene todas las observaciones que forman parte del DR previo,
agregando los datos recabados durante el periodo de observación para el DR
actual. Cada DR cae bajo una fase de operaciones de SDSS, delimitado tanto por
las fechas de observaciones como por los instrumentos y tipos de datos
disponibles. Para el periodo operacional de
\hyperref[muestra:sec:gaia:dr2]{GDR2} el catálogo más actual de SDSS era el DR9
publicado como parte de SDSS-III
\footnote{\url{https://www.sdss3.org/index.php}}. Esta tercera fase fue marcada
por una gran mejora del equipo espectroscópico, instalando nuevos instrumentos
con los cuales pudieron analizar la dinámica de nuestra Galaxia, al igual que
otras galaxias y planetas gaseosos extra-solares. 



\section{Szkody, et al. (2002): Cataclysmic Variables from the Sloan Digital Sky Survey} \label{muestra:szkody2002}

Con el lanzamiento del SDSS, Szkody y su equipo reconocieron una nueva área de
oportunidad para expandir la población de variables cataclísmicas (VCs)
conocidas en la Galaxia. De interés particular son aquellos sistemas que más se
aproximan al periodo mínimo según los modelos evolutivos de las VCs; estos
objetos llegan a magnitudes fuera del alcance de la mayoría de los telescopios
usados hasta este entonces, por lo cual no han sido el objetivo de estudio en la
literatura. Partiendo de SDSS Szkody y colaboradores iniciaron una búsqueda de
VCs tenues, con la expectativa de capturar una muestra representativa de
variables cataclísmicas en nuestra galaxia, en particular obteniendo muestras de
poblaciones históricamente imperceptibles a nuestros instrumentos.

Para restringir los sistemas que buscar, Szkody y colaboradores aplicaron un
criterio de color basado en el trabajo de \citeyearparen{krisciunas1998SdssCriteria}, en
el cual lograron determinar concentraciones de diferentes tipos de objetos
utilizando diagramas de color-color. A pesar de haber hecho estas observaciones
antes del año de lanzamiento de SDSS, Krisciunas y colaboradores lograron
obtener observaciones utilizando equipo cuyas características se asemejan a las
de los instrumentos utilizados para SDSS. Partiendo de estos resultados, Szkody
y colaboradores determinaron criterios en las regiones azules y rojas del
espectro, cuyos valores se encuentran en las  
\refequations{muestra:szkody2002:criterioeqs}.

\begin{equation}
	\begin{split}
		& u^* - g^* < 0.45 \\
		& g^* - r^* < 0.7 \\
		& r^* - i^* > 0.30 \\
		& i^* - z^* > 0.4
	\end{split}
	\label{muestra:szkody2002:criterioeqs}
\end{equation}

Una vez recabada la muestra de candidatas a observar, Szkody y colaboradores
confirmaron su estatus como variables cataclísmicas basado en los espectros
obtenidos del SDSS desde el \textit{Apache Point Observatory}. Estos datos los
complementaron con observaciones de espectrografía con el telescopio de 3.5m en
el \textit{Apache Point Observatory} y observaciones fotométricas utilizando el
telescopio de 0.76m en el \textit{Manastash Ridge Observatory} de la Universidad
de Washington. En total identificaron 22 sistemas como variables cataclísmicas,
incluyendo 3 objetos previamente estudiados e identificados como tal. Presentan
la concentración de los objetos en el diagrama color-color, vistos en la
\reffigure{szkody2002ColorColorVCs}. 

\begin{figure}[!ht]
	\centering
	\includegraphics[scale=0.4]{Muestra/Secciones/Figures/gr-ri_Szkody2002.png}
	\includegraphics[scale=0.4]{Muestra/Secciones/Figures/ri-iz_Szkody2002.png}
	\includegraphics[scale=0.4]{Muestra/Secciones/Figures/ug-gr_Szkody2002.png}

	\caption{Variables cataclísmicas identificadas y observadas por Szkody y
		colaboradores (círculos negros fuertes). Se puede apreciar la separación
		de las variables cataclísmicas del locus estelar, vista en los puntos
		negros. \citeyearparen{szkody2002CvSearchSdss}}
	\label{szkody2002ColorColorVCs}
\end{figure}
\section{Criterios de Selección} \label{muestra:crit_seleccion}

% Para identificar estas candidatas a variables cataclísmicas utilizamos el
% catálogo de \hyperref[muestra:sec:gaia]{\gaia} para obtener una amplia muestra
% fotométrica de estrellas dentro de la Galaxia. Gracias a la alta precisión y
% sensibilidad de los instrumentos de \gaia es posible observar estrellas que no
% han sido bien documentadas en la literatura. Para esto utilizamos la base de
% datos dinámica de SIMBAD\footnote{\url{http://simbad.cds.unistra.fr/simbad/}}
% para identificar estrellas cuya clasificación sea de interés para nuestra
% investigación. Estas las priorizamos en base a la cantidad de datos disponibles
% en la literatura; aquellos sistemas con la menor cantidad de referencias en la
% literatura (obtenidas de SIMBAD) tienen una mayor prioridad que aquellos
% objetos con varios estudios publicados. 

Este trabajo tiene como objetivo realizar una campaña de observación para un
sistema pobremente estudiado, con el propósito de confirmar su estatus como
variable cataclísmica o como una binaria eclipsante, dependiendo del sistema.
Para esto, se implementó un proceso para separar e identificar objetos de
interés para observar desde el Observatorio Astronómico Universitario en
Iturbide. A continuación se describe los aspectos técnicos importantes de la
búsqueda. El código completo se encuentra en la carpeta
\href{https://github.com/KnightIV/UANL_MAPTA_Observaciones/tree/main/obsrv_plan}{\code{obsrv\_plan}},
cuyo punto de entrada se ubica en el script
\href{URLhttps://github.com/KnightIV/UANL_MAPTA_Observaciones/blob/main/obsrv_plan/main.py}{\code{main.py}}.

\subsection{Búsqueda en Gaia}  \label{muestra:crit_seleccion:busqueda_fotometrica}

Para obtener la muestra inicial de objetos de interés acudimos a la base de
datos de Gaia. Tal como es descrito en la sección \ref{muestra:sec:gaia} la
selección de objetos fue llevada a cabo dentro del \textit{Gaia Archive}
utilizando su interfaz de ADQL. Sin embargo, los criterios definidos por Szkody
y colaboradores solo fueron definidos para el sistema fotométrico de SDSS; para
poder utilizar estos primero se llevó a cabo una conversión de las magnitudes
reportadas en el catálogo de Gaia a magnitudes en los pasa bandas de SDSS. Esta
conversión se llevó a cabo utilizando las siguientes relaciones definidas en la
documentación de Gaia DR3 \citet{gdr3ReleaseDocumentation}, como se puede
ver en la figura \ref{gdr3SdssConversionGraphs}. Partiendo de estas magnitudes
transformadas se aplicó los criterios definidos en
\citet{szkody2002CvSearchSdss}. Sin embargo, solo dos de los 4 indices de color
se pueden aplicar a la muestra de Gaia; no están definidas transformaciones para
las bandas $u$ ni $z$ de SDSS, ya que estas abarcan longitudes de onda más
extremas que las observadas por Gaia. El query de ADQL ejecutada se puede
encontrar en el apéndice \ref{apendice:gaiaAdql}. Se obtuvieron en total más de
\num{3630000} fuentes, el cual representa un 0.2\% de los 1,811,709,771 objetos
reportados en el DR3 de Gaia. Un query similar fue ejecutado en la base de datos
de Gaia para el segundo Data Release (DR2) (apéndice \ref{apendice:gaiaAdql:dr2}). 

% TODO agrega resultados de ambas queries. quizás en el apéndice?

\begin{figure}[!ht]
	\centering
	\includegraphics[scale=0.18]{Muestra/Secciones/Figures/Gaia-SDSS-Transform-g.png}
	\includegraphics[scale=0.18]{Muestra/Secciones/Figures/Gaia-SDSS-Transform-i.png}
	\includegraphics[scale=0.18]{Muestra/Secciones/Figures/Gaia-SDSS-Transform-r.png}

	\caption{Relación empírica entre las magnitudes reportadas en GDR3 y SDSS12. Las relaciones están dadas para 3 de las 5 bandas de SDSS12, debido a las diferencias entre las pasa bandas de Gaia y SDSS. \citet{gdr3ReleaseDocumentation}}
	\label{gdr3SdssConversionGraphs}
\end{figure}

\subsection{Selección de Objetos Observables} \label{muestra:crit_seleccion:objetos_observables}

La ubicación en la bóveda celeste de un sistema candidata juega un papel
importante en la viabilidad de una campaña de observación desde el OAU. Esto
determina si un objeto es visible desde la locación geográfica del observatorio
durante las fechas de observación; de otra manera sería imposible apuntar un
telescopio al sistema. Para realizar esta tarea se utilizaron los módulos de
\code{astroplan} \citet{astroplan} y \code{astropy} \citet{astropy}, aplicando
el algoritmo a los objetos resultados de la búsqueda en la base de datos de
Gaia. El código responsable se encuentra en el archivo
\href{https://github.com/KnightIV/UANL_MAPTA_Observaciones/blob/main/obsrv_plan/gaia/observable_targets.py}{\code{observable\_targets.py}}.

% TODO: agrega resultados de cuantos objetos en total fueron elegidos?

\subsection{Búsqueda en SIMBAD} \label{muestra:crit_seleccion:busqueda_simbad}

% TODO: pregunta como referir al trabajo (presente o pasado? o hasta futuro?)
Una vez obtenidos los objetos de interés de la selección de objetos visibles se
utilizó la base de datos de
SIMBAD\footnote{\url{http://simbad.cds.unistra.fr/simbad/}} para restringir los
objetos de interés a un tamaño manejable, con el objetivo de obtener un sistema
clasificado como variable cataclísmica, binaria eclipsante, o candidata a alguna
de estas clasificaciones. Este sistema será estudiado desde el
\textbf{Observatorio Astronómico Universitario} en Iturbide, Nuevo León. Por lo
tanto, un requisito para este trabajo de maestría es que este sistema sea uno
con una cantidad minima de estudios antecedentes; el estudio del sistema
dependerá en gran parte de la curva de luz obtenida de las observaciones. Esta
búsqueda śe llevó a cabo utilizando el API de SIMBAD, el cual acepta mensajes y
encuestas por HTTP. El código relevante a esta búsqueda se encuentra en
\href{https://github.com/KnightIV/UANL_MAPTA_Observaciones/blob/main/obsrv_plan/simbad/retrieve_vots.py}{\code{retrieve\_vots.py}}.
A pesar de no haber obtenido una muestra significativa de candidatas a binarias
eclipsante se identificó un sistema de interés.

% TODO: agrega histograma de SIMBAD
\chapter{Observaciones}

El segundo objetivo principal de este trabajo de investigación fue realizar una
campaña de observación al objeto de interés \atoObjId. Desde el
\textbf{Observatorio Astronómico Universitario} (\textbf{OAU}) en Iturbide, N.L.
se midió el brillo del sistema durante 9 noches de observaciones sin filtro, del
cual utilizando la técnica de fotometría diferencial se obtuvo una curva de luz
del sistema. Los datos observacionales producidos de este estudio se utilizan
como un conjunto complementario a las curvas fotométricas del objeto descritas
en los capítulos anteriores; a pesar de no tener información del color del
sistema las observaciones realizadas desde el OAU son de las más recientes
realizadas\textemdash junto a los datos de ZTF\textemdash los cuales ofrecen una
vista al comportamiento y estado actual de \atoObjIdNoSpace.

\section{Observatorio Astronómico Universitario - Iturbide}

El \textbf{Observatorio Astronómico Universitario - Iturbide} (el cual de ahora
en adelante será referido como el OAU), ubicado en el cerro Picacho en el
municipio de Iturbide, Nuevo León, es un nuevo sitio dedicado a la observación
astronómica, equipado para realizar observaciones del Sol, monitoreo de basura
espacial, y la observación de objetos variables, como los sistemas binarios o
asteroides. A continuación se describe el equipo utilizado; como software de
control se utilizó \textbf{Nighttime Imaging 'N'
Astronomy}\footnote{\myurl{https://nighttime-imaging.eu}} (\textbf{NINA}), el cual
permita consolidar el control de todas las componentes mecánicas en una sola
aplicación. 

% TODO: look for mount documentation
El telescopio utilizado para hacer las observaciones del sistema fue el tubo
óptico \textbf{CDK20} de \textbf{PlaneWave
Instruments}\footnote{\myurl{https://planewave.com/product/cdk20-ota/}} con un
número $f/6.8$. Este telescopio de diseño \textit{Dall-Kirkham corregido} cuenta
con un grupo de lentes frente al espejo esférico secundario, el cual resta los
efectos de la aberración esférica presente en otras configuraciones de espejos
primarios y secundarios, resultando en una imagen más nítida. Este instrumento,
combinado con una montura ecuatorial \textbf{Orion HDX110}, nos permite una
vista clara de la bóveda celeste a \ang{30} arriba del horizonte, con capacidad
de observar objetos tenues más allá de 17 magnitudes.

% TODO: add filter wheel info
El CCD usado para obtener las imágenes fue el
\textbf{QHY174GPS}\footnote{\myurl{https://www.qhyccd.com/qhy174gps-imx174-scientific-cooled-camera/}}.
Este CCD cuenta con una resolución de $1920 \times 1200$ pixeles. Para reducir
el ruido térmico tiene un mecanismo de enfriamiento termoeléctrico, el cual lo
puede enfriar a una temperatura de -\ang{40} C bajo la temperatura ambiente. 
\section{Fotometría}

Para este trabajo se realizó una campaña de observación durante los últimos
meses del 2022, con la finalidad de observar el sistema durante una fase orbital
completa. Las fechas y duraciones de cada día de observación se encuentra en la
\reftable{observationSchedules}. Varias de las noches de observaciones sufrieron
de pobres condiciones del sitio, las cuales llegaron a afectar la calidad de las
observaciones; tanto las condiciones meteorológicas como contratiempos causados
por el equipo en si causaron interrupciones en las exposiciones continuas y
perturbaciones en las imágenes mayores a las que se puede corregir en el
procesamiento de datos. Esto es esperado en un observatorio en proceso de
desarrollo; a pesar de los problemas técnicos, se pudieron obtener datos de
calidad aceptable. Los vientos fuertes fueron un gran obstáculo debido a la nula
protección del CCD montado al telescopio, ya que cualquier perturbación por
parte del viento resultaba en una imagen defectuosa. Dado que las observaciones
de este proyecto se realizaron en el 2022 solo el viento sigue siendo un
problema para observaciones posteriores; todavía no se implementa una cúpula
para proteger el telescopio del viento, pero el auto-enfocador y guiador están
funcionando de manera apropiada.

% TODO: style table
\begin{table}[!ht]
	\centering
	\begin{tabular}{|c|c|c|c|}
		\hline
		\thead{Fecha (UTC)} & \thead{HJD Inicio +\textbf{\num{2459000}}} & \thead{Tiempo Expocisiones} & \thead{Duración} \\
		\hline
		2022-10-22 & 874.67 & $111 \cdot 60$ s & 2.67 h \\
		\hline
		2022-10-23 & 875.57 & $159 \cdot 60$ s & 5.28 h \\
		\hline
		2022-10-28 & 880.77 & $124 \cdot 60$ s & 2.11 h \\
		\hline
		2022-11-06 & 889.58 & $125 \cdot 60$ s & 4.25 h \\
		% \hline
		% 2022-11-26 & 910.72 & $56 \cdot 60$ s & 1.55 h \\
		\hline
		2022-12-07 & 920.55 & $231 \cdot 60$ s & 5.25 h \\
		\hline
		2022-12-08 & 921.57 & $138 \cdot 60$ s & 4.25 h \\
		\hline
		2022-12-09 & 922.54 & $127 \cdot 60$ s & 4.97 h \\
		\hline
		2022-12-10 & 923.54 & $129 \cdot 60$ s & 5.44 h \\
		\hline
		2022-12-11 & 924.53 & $122 \cdot 60$ s & 2.20 h \\
		\hline

	\end{tabular}
	\caption{Bitácora de fechas de observaciones fotométricas desde el OAU.}
	\label{observationSchedules}
\end{table}

\subsection{Estrellas de Comparación}

Para determinar la magnitud diferencial de un objeto se necesita una estrella de
comparación dentro del campo de la imagen de ciencias. Una manera de
encontrar estrellas de referencias adecuadas es utilizando el Variable Star
Plotter\footnote{\url{https://app.aavso.org/vsp/}} de la \textit{AAVSO}; sin
embargo, debido al pequeño campo de visión de nuestras imágenes
(aproximadamente $1^{\prime}$ de largo), no se encuentra alguna estrella
standard registrada. Por lo tanto para realizar la fotometría diferencial solo
es necesario tener estrellas de comparación con un número de cuentas (flujo
integrado) similar al objeto de interés. El campo visible de \atoObjId
utilizando el equipo de OAU se puede ver en la \reffigure{figuraCcdCampo}.

\begin{figure}[!ht]
	\centering
	\includegraphics[scale=0.5]{Observaciones/Secciones/Figures/Figura Campo Observado.png}
	\caption{Imagen del campo de \atoObjId marcado con la etiqueta $1$ en, 
	junto a las estrellas de referencia usadas en la fotometría diferencial 
	marcadas en anillos rosas marcados con etiquetas del $2$ al $7$.}
	\label{figuraCcdCampo}
\end{figure}

\subsection{Procesamiento de Imágenes}

La limpieza de las imágenes incluyó la corrección de bias, darks, y flats por
medio de imágenes de calibración, las cuales fueron tomadas cada noche de
observación. Esto fue realizado utilizando tareas standard de IRAF
[\citeyearparen{tody_iraf_1986}]. Cada imagen fue revisada manualmente para
determinar si era de suficiente calidad para hacer una medición adecuada. Se
analizó la forma que proyecta el objeto en la imagen del CCD; solo las imágenes
cuyo perfil se aproxima a un circulo fueron aceptadas para realizar el proceso
de fotometría diferencial.

\begin{figure}[!ht]
	\centering
	\includegraphics[scale=0.4]{Observaciones/Secciones/Figures/Figura Pixel Perfil Contorno.png}
	\caption{Isógrama de las cuentas registradas del flujo de \atoObjId para un
	corte de $14 \times 14$ pixeles, donde el color más claro denota una mayor
	cantidad de cuentas.}
	\label{figuraPixelContorno}
\end{figure}

Una vez que las mediciones fueran corregidas, fue necesario trasladar los datos
dentro de las imágenes para que \atoObjId quede en el centro del campo,
facilitando la fotometría por apertura. Utilizando el script
\href{https://github.com/KnightIV/UANL_MAPTA_Observaciones/blob/main/analisis/iturbide/shift_images.py}{\code{shift\_images.py}}
se ejecutó una tarea de \textit{plate solving} para cada imagen calibrada; el
proceso de \textit{plate solve}, llevado a cabo utilizando el programa
\textit{Astrometry} [\citeyearparen{astrometry}], toma como referencia estrellas
dentro del campo de la imagen comparando contra una base de datos pre-definida
para determinar las coordenadas físicas que corresponden a una imagen. Esta
información va encapsulada dentro de los metadatos del archivo FITS, conocido
como \textbf{World Coordinate System} (\textbf{WCS}). Una vez que una imagen sea
resuelta se puede proyectar a las coordenadas de otra imagen, posicionando la
estrella variable en una posición única dentro del encuadro de cada imagen,
facilitando el uso de coordenadas en pixeles para definir las aperturas de
medición fotométrica.

Utilizando las imágenes proyectadas se obtuvo el brillo del objeto utilizando
la tarea \code{qphot} de IRAF, dando como resultado magnitudes instrumentales
del sistema. La apertura en tamaño de pixeles se definió en base al ancho a
media altura (\textit{full-width at half-maximum}) de \atoObjId para cada
imagen, utilizando la tarea \code{imexam} de IRAF. Este se usó como el radio de
la apertura circular, junto a una apertura anular para sustraer el brillo de
fondo del cielo, el cual se utilizó para las 7 estrellas marcadas en la
\reffigure{figuraCcdCampo}. Las magnitudes instrumentales medidas se pueden ver
en la \reffigure{figuraCurvaLuzInstrumentalTodas}. A pesar de que todos los
objetos muestran un comportamiento variable, esto no es una característica
intrínseca de todos los sistemas. En particular, la estrella de comparación se
considera que emite un flujo relativamente constante a comparación del sistema
variable \atoObjIdNoSpace; por lo tanto, las variaciones vistas en su curva de
luz se deben a perturbaciones atmosféricas e instrumentales, lo cual nos permite
eliminar estos efectos por medio de la fotometría diferencial.

\begin{figure}[!ht]
	\centering
	\includegraphics[scale=0.4]{Observaciones/Secciones/Figures/Figura Magnitud Instrumental Todas.png}
	\caption{Magnitudes instrumentales reportada por IRAF mediante la técnica de
	fotometría de apertura. De los 7 objetos resaltados en la
	\reffigure{figuraCcdCampo} se distinguen el objeto principal \atoObjId con
	la etiqueta $1$, y la estrella de referencia utilizada para determinar la
	magnitud diferencial de \atoObjIdNoSpace. }
	\label{figuraCurvaLuzInstrumentalTodas}
\end{figure}

\subsection{Fotometría Diferencial}
Para obtener una magnitud diferencial de \atoObjId se necesita de una estrella
de referencia de la cual se observe un flujo constante. Aparte del requisito de
ser una fuente constante, una estrella de referencia ideal sería del mismo color
que el sistema variable \atoObjIdNoSpace; sin embargo, en casos como el de este
estudio se puede despreciar este último requisito debido al bajo número de
opciones disponibles dentro del campo visible. Del campo visible en la
\reffigure{figuraCcdCampo} se eligió el objeto número $5$ como la estrella de
referencia. Su curva de luz instrumental se puede ver a mayor detalle en la
\reffigure{figuraCurvaLuzInstrumentalReferencia}. La magnitud diferencial $m_d$
de \atoObjId se obtiene restando la magnitud instrumental del objeto de
referencia 5 $m_{\mathrm{inst, 5}}$ de la magnitud instrumental medido de
\atoObjId $m_{\mathrm{inst, \atoObjId}}$ utilizando la
\refequation{ecuacionMagnitudDiferencial}, cuya curva de luz resultante se puede
ver en la \reffigure{figuraIturbideAtoLightCurve}. Las imágenes calibradas se
encuentran como un \textit{Release} dentro del repositorio de GitHub este
proyecto de investigación, junto a los datos de la fotometría de
apertura.\footnote{\url{https://github.com/KnightIV/UANL_MAPTA_Observaciones/releases/tag/data}}

\begin{eqfloat}[!ht]
	\begin{equation}
		m_d = m_{\mathrm{inst, \atoObjId}} - m_{\mathrm{inst, 5}}
	\end{equation}
	\blankcaption
	\vspace{-0.5em}
	\label{ecuacionMagnitudDiferencial}
\end{eqfloat}

\begin{figure}[!ht]
	\centering
	\includegraphics[scale=0.44]{Observaciones/Secciones/Figures/Figura Magnitud Instrumental Estrella Referencia.png}
	\caption{Magnitud instrumental de la estrella de referencia utilizada para
	determinar la magnitud diferencial de \atoObjIdNoSpace. La variabilidad en
	la curva de luz se atribuye a factores extrínsecos del sistema.}
	\label{figuraCurvaLuzInstrumentalReferencia}
\end{figure}

\begin{figure}[!ht]
	\centering
	\xincludegraphics[scale=0.42, label=\textbf{(a)}, labelbox=true, pos=nw, fontsize=\large]{Observaciones/Secciones/Figures/Full LC.png}
	\xincludegraphics[scale=0.34, label=\textbf{(b)}, labelbox=true, pos=nw, fontsize=\large]{Observaciones/Secciones/Figures/Hour Sync LC.png}
	\caption{Magnitud diferencial de \atoObjIdNoSpace. \textbf{(a)} Curva de luz
	completa. \textbf{(b)} Curva de luz segmentada por día, con la cual se logra
	apreciar la cadencia de observación, viendo como cada día se logra observar
	una fase diferente del sistema.}
	\label{figuraIturbideAtoLightCurve}
\end{figure}
\part{Metodología y Análisis de Datos} \label{metodologia}

\section{Análisis del Periodo Orbital} \label{metodologia:analisisperiodo}

Una de las propiedades más importantes presente en la curva de luz de una
binaria eclipsante es su \textbf{periodo orbital}. Partiendo del periodo orbital
es posible presentar los datos observacionales en el espacio fase en vez de
tiempo, el cual nos permite ajustar modelos analíticos para determinar ciertas
propiedades del sistema binario. Dada una curva de luz se puede encontrar el
periodo orbital usando \textbf{periodogramas}: herramientas utilizadas para
generar un espectro de potencias para una serie de tiempo periódica. Para series
de tiempo cuyo muestreo no es uniforme en el tiempo (como es común de
observaciones astronómicas) se utiliza el periodograma \textbf{Lomb-Scargle},
derivado de la transformada de Fourier \brakcite{understandingLombScargle}.
Usando un mallado suficientemente fino para explorar el espacio de frecuencias
se puede encontrar la frecuencia de mayor potencia, indicando el periodo orbital
del sistema. El espectro de frecuencias se encuentra en la figura
\ref{periodogramaLSFrecs}. El código para determinar el periodo orbital se ubica
en el Notebook
\href{https://github.com/KnightIV/UANL_MAPTA_PlanObservaciones/blob/0b571b7377a76e0360d5e142924cc964194ace8b/analisis/phoebe_model/estimations/periodogram.ipynb}{\code{periodogram.ipynb}}.

% TODO: Iturbide LC en fase

\begin{figure}[!ht]
	\centering
	\includegraphics[scale=0.55]{Metodologia/Secciones/AnalisisPeriodo/Figures/LS Power Spectrum.png}
	
	\caption{Espectro de frecuencias de las curvas de luz fotométricas de
	\atoObjIdNoSpace, usando los datos recabados de Iturbide y de Gaia. Esta fue
	generada usando el periodograma dentro de los estimadores de PHOEBE, el cual
	utiliza el periodograma Lomb-Scargle en Astropy.} 
	\label{periodogramaLSFrecs}
\end{figure}

Dado este espectro de frecuencias encontramos que el periodo orbital yace en la
segunda armónica de la frecuencia principal. Esto se debe a los requisitos para
analizar una curva de luz de un sistema binario eclipsante; estos muestran dos
valles en el espacio fase, las cuales corresponden a las etapas en la curva de
luz en las que se observan eclipses en el sistema. Esto es necesario para poder
modelar la curva de luz en fase como una Gaussiana doble, el modelo aceptado
para una binaria eclipsante. %TODO: agrega bibliografía
Utilizando la segunda armónica de la frecuencia de más alta potencia se puede
ver esta forma esperada de la curva de luz, como se puede ver en la figura
\ref{gaiaIturbidePhaseFold}. El periodo orbital encontrado es de 8.0049 horas.

\begin{figure}[!ht]
	\centering
	\includegraphics[scale=0.8]{Metodologia/Secciones/AnalisisPeriodo/Figures/Gaia Phase-Folded.png}
	\includegraphics[scale=0.8]{Metodologia/Secciones/AnalisisPeriodo/Figures/Iturbide Phase-Folded.png}

	\caption{Curvas de luz de Gaia e Iturbide en espacio fase dado un periodo orbital de 8.0049 horas.}
	\label{gaiaIturbidePhaseFold}
\end{figure}
\chapter{Normalización de Flujos y Preservación de Color} \label{Metodologia:NormalizacionFlujos}

Las curvas de luz obtenidas de los catálogos Gaia y ZTF al igual que las
observaciones realizadas en el OAU permiten el estudio de \atoObjId ajustando un
modelo de un sistema binario eclipsante utilizando PHOEBE. Sin embargo, PHOEBE
trabaja directamente con flujos; al calcular el modelo hacia adelante PHOEBE
integra las intensidades calculadas para cada elemento superficial de ambas
estrellas e integra la radiación emergente en la dirección del observador para
obtener una medición del flujo en unidades de $\mathrm{W} \mathrm{m}^{-2}$. Los
datos obtenidos del catálogo de Gaia vienen con el flujo registrado por el
satélite de cada tránsito del objeto por su campo de visión\textemdash este
viene reportado en unidades de $\mathrm{e}^{-} \mathrm{s}^{-1}$, el cual se
puede utilizar directamente en PHOEBE en el caso de no necesitar conocer la
luminosidad absoluta de las componentes estelares. 

Los datos recabados del catálogo de ZTF no reportan alguna medición directa del
flujo recibido\textemdash estos solo reportan las magnitudes y sus errores
obtenidas de su procesamiento fotométrico. El procesamiento de las imágenes en
cada filtro descrito por
\citeyearparen{masci_ztf_data_processing_products_archive_2018} da como
resultado curvas de luz calibradas con estrellas de calibración del catálogo
Pan-STARRS1. Aunque no es posible obtener una medición del flujo recibido sin
conocer el flujo $f_0$ que corresponde a una magnitud $m_0$, es posible obtener
curvas de flujos normalizados que preservan la información del color del sistema
ofrecido por la fotometría multibanda. Para cada punto de magnitud $m_p$ para la
pasabanda $p$ se aplica la siguiente transformación para obtener el flujo
normalizado:

\begin{eqfloat}[!ht]
	\centering
	\begin{equation}
		f_p = 10^{-\frac{2}{5} (m_p - m_0)}
	\end{equation}
	\blankcaption
	\label{ecuacionNormalizarFlujos}
\end{eqfloat}

Donde $m_0$ es la magnitud de referencia. En vez de escoger una magnitud $m_0$
para cada pasabanda se elige solo un valor de $m_0$ el cual se aplica a ambas
curvas de ZTF, dando como resultado el flujo relativo obtenido del sistema en
base a su distribución de energía espectral (SED). Este dato permite el ajuste
de la temperatura efectiva del sistema sin necesidad de acudir a una estimación
a priori basado en relaciones analíticas o estadísticas con respecto a otros
parámetros del modelo
(\citetbooksection{kallrath_eclipsing_binary_modelling_2009}{5.1.2.2}). A la
vez, al tener una manera concreta de evaluar de manera directa la temperatura
efectiva del sistema es posible conocer la incertidumbre utilizando solo datos
fotométricos. Para obtener el flujo normalizado de las curvas de ZTF se utilizó
la magnitud del sistema en la fase 0.25 en el pasabanda ZTF:g. El resultado se
puede ver en la \reffigure{figuraNormFlujosCurvas}. El código donde se realizó
este procedimiento se encuentra en el Notebook
\href{https://github.com/KnightIV/UANL_MAPTA_Observaciones/blob/main/analisis/ztf/light-curve-processing.ipynb}{\code{light-curve-processing.ipynb}}.

Debido a que la curva de luz de las observaciones hechas en el OAU en Iturbide,
N.L. son magnitudes diferenciales, no es posible determinar el flujo real
incidente a la CCD. Por lo tanto se adopta el mismo procedimiento hecho para las
curvas de luz de ZTF. Utilizando la \refequation{ecuacionNormalizarFlujos} los
flujos normalizados se obtuvieron usando la magnitud en fase 0.25 como el valor
de $m_0$, el código presente en el Notebook
\href{https://github.com/KnightIV/UANL_MAPTA_Observaciones/blob/main/analisis/period-analysis/periodogram.ipynb}{\code{periodogram.ipynb}}.
La curva de luz resultante se puede ver en la
\reffigure{figuraNormFlujosCurvas}. Como se mencionó al principio del capítulo,
no fue necesario tratar las curvas obtenidas de Gaia DR3 de esta manera, debido
a que los datos obtenidos de la fotometría de
época\footnote{\url{https://gea.esac.esa.int/archive/documentation/GDR3/Gaia_archive/chap_datamodel/sec_dm_photometry/ssec_dm_epoch_photometry.html}}
ya reporta el flujo medido por Gaia. 

\begin{figure}[!ht]
	\centering
	\includegraphics[scale=0.4]{Metodologia/Secciones/NormalizacionFlujos/Figures/ZTF Normalized Flux.png}
	\includegraphics[scale=0.4]{Metodologia/Secciones/NormalizacionFlujos/Figures/Iturbide Normalized Flux.png}

	\caption{Flujo en fase de las curvas de luz obtenidas para \atoObjIdNoSpace.
	De estos dos conjuntos de datos solo ZTF contiene información del color del
	sistema, y por ende una cantidad medible del cual determinar la temperatura
	efectiva del sistema.}
	\label{figuraNormFlujosCurvas}
\end{figure}
\section{Modelo Computacional} \label{metodologia:modelocomputacional}

Usando todas las curvas de luz disponible para el sistema \atoObjId --- tanto de
Gaia como los datos recabados de Iturbide --- se puede generar un modelo
computacional cuyas propiedades físicas pueden adecuadamente explicar los datos
observacionales. Este método al final daría como resultado una \textit{solución
fotométrica} del sistema. En el mejor de los casos, esta solución muestra un
valor satisfactorio de ajuste a los datos observacionales. A continuación se
plasma el proceso que se llevó a cabo para llegar a una solución fotométrica del
sistema \atoObjIdNoSpace; esta solución no es única en el sentido que otra
combinación de parámetros podría llegar a las mismas conclusiones.

\subsection{Estimaciones Iniciales}
\label{metodologia:modelocomputacional:estimacionesiniciales}

Una vez determinado el periodo orbital del sistema se puede empezar un estudio
de la morfología de las curvas de luz en fase. PHOEBE para facilitar esto ofrece
distintos métodos para generar los las primeras estimaciones de parámetros
físicos del sistema. El estimador \textbf{EBAI-KNN} para estimar los siguientes
parámetros: el \textit{tiempo de conjunción superior} (\code{t0\_supconj}), la
\textit{razón de temperaturas} (\code{teffratio}), la \textit{inclinación
orbital} (\code{incl@binary}), el \textit{factor de relleno}
(\quotes{\textit{fillout factor}} en inglés, \code{fillout\_factor}), y la
\textit{razón de masas} (\code{q}). A pesar que dentro de PHOEBE estén
implementados estimadores adicionales, solo se puede aplicar el
\textbf{EBAI-KNN} estimador; esto se debe a que el modelo del sistema del que
parte este trabajo corresponde al de una binaria en contacto (elegido por la
morfología aparente de la curva de luz de Iturbide).

Dentro del Jupyter Notebook
\href{https://github.com/KnightIV/UANL_MAPTA_PlanObservaciones/blob/main/analisis/phoebe_model/estimations/ebai-default.ipynb}{\code{ebai-default.ipynb}}
se puede encontrar el código con el que se llevaron a cabo las pruebas de
estimación de parámetros. El estimador \textbf{EBAI-KNN} puede que obtenga
diferentes soluciones del sistema dependiendo de la curva de luz utilizada; por
lo cual se esperaba que obtuviera diferentes resultados dependiendo de la curva
de entrada. Para obtener un panorama completo de las posibles soluciones
fotométricas so ejecutaron varios estimadores de PHOEBE, cada uno operando sobre
una diferente combinación de curvas de luz; se corrió un estimador por cada
curva de luz individual, al igual que unos estimadores que tuvieron de entrada
una combinación de curvas de luz de Gaia e Iturbide. El experimento completo
junto a sus curvas de luz sintéticas correspondientes se pueden ver en el
Notebook antedicho, acompañado de las gráficas resultantes de cada estimador.

\subsubsection{Elección de Modelo Inicial}
\label{metodologia:modelocomputacional:estimacionesiniciales:eligiendomodeloinicial}

Una consideración importante en el proceso de modelación computacional es la
existencia de diferentes soluciones fotométricas dado un mismo conjunto de
datos. Esto se debe a la ortogonalidad de los parámetros en el sistema; dos o
más parámetros pueden estar en un estado de degeneración, donde existe una
relación lineal entre estos, lo cual significa que no existe una solución única
correcta del sistema. Para decidir entre los varios estimadores se tomó como
criterio de selección el ajuste del "forward model" a los datos mediante la
estadística $\chi^2$. Estos se pueden ver en la figura \ref{chiSqrdFigure}.
Partiendo de la medición del ajuste de cada modelo se ve que
\code{ebai\_knn\_raw} y \code{ebai\_knn\_lc\_iturbide\_raw} son los que mejor se
acoplan a los datos observacionales. La optimización de parámetros se llevó a
cabo partiendo de las estimaciones de \code{ebai\_knn\_raw}, el cual de entrada
recibió las cuatro curvas de luz de este trabajo (3 de Gaia, 1 de Iturbide). El
resultado inicial del modelo se puede ver en la figura
\ref{ebaiKnnRawEstimateModel}, junto a los parámetros del modelo en la tabla \ref{ebaiKnnInitialEstimationsValues}.

\begin{figure}[!ht]
	\centering
	\includegraphics[scale=0.43]{Metodologia/Secciones/ModeloComputacional/Figures/EstimadoresChiResultados.png}
	
	\caption{Resultados de $\chi^2$ de los modelos sintéticos generados
	utilizando los parámetros de los estimadores. Cada estadística fue calculada
	con respecto a todos los datos observacionales disponibles, sin importar las
	combinaciones de curvas de luz utilizadas para hacer la estimación.
	\code{raw\_model} corresponde al modelo inicial que ofrece PHOEBE a través de
	la función \code{phoebe.default\_contact\_binary()}.} 
	\label{chiSqrdFigure}
\end{figure}

\begin{figure}[!ht]
	\centering
	\includegraphics[scale=0.5]{Metodologia/Secciones/ModeloComputacional/Figures/ebai_knn_raw_estimacion.png}
	
	\caption{Modelos sintéticos del modelo utilizando los parámetros estimados
	por \code{ebai\_knn\_raw\_solver} junto a los residuos en los flujos para
	cada curva de luz. Estos modelos fueron sintetizados utilizando un factor de
	escala de flujos flexible, utilizando la opción \code{pblum\_mode =
	"dataset\_scaled"}, el cual nos permite analizar la morfología del modelo
	sintético sin considerar por ahora el efecto en la escala de la curva de
	parámetros relacionados con la luminosidad de cada componente, como las
	temperaturas absolutas de ambas estrellas. Estos parámetros son ajustados en
	los siguientes pasos de afinación del modelo.}
	\label{ebaiKnnRawEstimateModel}
\end{figure}

% TODO: style table
\begin{table}[!ht]
	\centering
	\begin{tabular}{|l|l|}
		\hline
		% \rowcolor{blue}
		\thead{Parámetro} & \thead{Valor} \\
		\hline
		\code{t0\_supconj@binary} & 0.06841 d \\
		\hline
		\code{teffratio@binary} & 0.99560 \\
		\hline
		\code{incl@binary} & 1.25572 rad \\
		\hline
		\code{fillout\_factor@contact\_envelope} & 0.51640 \\
		\hline
		\code{q@binary} & 3.49495 \\
		\hline

	\end{tabular}
	\caption{Resultados adoptados de las estimaciones iniciales, utilizando el estimador cuyos datos de entrada fueron las cuatro curvas de luz disponibles. Las unidades de cada valor son especificadas excepto para los parámetros adimensionales.}
	\label{ebaiKnnInitialEstimationsValues}
\end{table}
\subsection{Optimización de Parámetros}

Como se puede ver en la \reffigure{ebaiKnnGaiaEstimateModel} el modelo inicial
de PHOEBE no se ajusta perfectamente bien a los dados datos observacionales. Los
residuos de los flujos no llegan a estar planos alrededor de 0; se alcanza a
apreciar un comportamiento oscilatorio en fase. Por lo tanto es necesario
ajustar estos parámetros para llegar al mínimo global en el espacio de
parámetros. PHOEBE ofrece varias herramientas partiendo de una estimación
inicial, haciendo uso de métodos numéricos para evaluar el espacio de parámetros
y llegar a un modelo con el mejor ajuste a los datos observacionales. 
\part{Resultados y Conclusiones}

\chapter{Consideraciones del Modelo de PHOEBE} \label{conclusion:consideraciones_phoebe}

El resultado final obtenido en el
\refthesischapter{metodologia:modelocomputacional} está sujeto a varias
consideraciones. En general, es imposible constreñir de manera adecuada varios
parámetros del modelo de un sistema binario estelar utilizando solo curvas
fotométricas resueltas en el tiempo, y aún menos con solo magnitudes
diferenciales. Para tener un modelo más completo se requiere datos
complementarios, como curvas de velocidades radiales para poder obtener el
semi-eje mayor en unidades reales y por ende las masas de ambas
componentes\textemdash cosa que es posible dentro de PHOEBE utilizando
estimadores y optimizadores adicionales a los empleados en este proyecto de
investigación. En este capítulo se plasman detalles particulares con el modelo
sintético derivado, incluyendo degeneraciones en el modelo y un comportamiento
multi-modal que considerar.

\section{Datos Espectroscópicos}

En el transcurso de este proyecto se buscó entre otras fuentes adicionales para
intentar encontrar observaciones espectroscópicas de \atoObjId que pudieran
constreñir otros parámetros del sistema, como la temperatura efectiva del
sistema (la cual se dejaría como parámetro fijo en vez de utilizar la diferencia
de color en el flujo de ZTF) o la masa de la estrella primaria, la cual no se
puede constreñir utilizando solo curvas de luz fotométricas. A pesar de las
capacidades avanzadas de PHOEBE, actualmente no tiene manera de generar un
espectro sintético con cual comparar a un espectro observado. En total se
obtuvieron 3 espectros independientes de \atoObjIdNoSpace : 2 de ellos fueron
proporcionados por la Dra. Paula Szkody de la Universidad de Washington, y 1
espectro obtenido del catálogo Gaia DR3.

\subsection{Apache Point Observatory}

Desde el \textit{Observatorio Apache Point} (\textit{Apache Point Observatory})
la Dra. Szkody logró observar a \atoObjId con el telescopio \textbf{ARC 3.5m}
con el espectrógrafo \textbf{KOSMOS}, un espectrógrafo de baja resolución ($R
\sim 2200$) con una rendija de 0.86 pulgadas en la posición \textit{Alta}, la
cual resulta en un rango de longitudes de onda de $4150 - 7050 \Angstrom$. La
información técnica de KOSMOS se encuentra en la documentación en línea de
KOSMOS\footnote{\url{https://www.apo.nmsu.edu/arc35m/Instruments/KOSMOS/userguide.html}}.
Ambos espectros se pueden ver en la \reffigure{figuraEspectrosApo}.

\begin{figure}[!ht]
    \centering
    \includegraphics[scale=0.4]{Conclusion/Figures/Figura APO Spectrum 2.png}
    \includegraphics[scale=0.4]{Conclusion/Figures/Figura APO Spectrum 1.png}
    \caption{Espectros tomados de \atoObjId por la Dra. Paula Szkody desde APO.}
    \label{figuraEspectrosApo}
\end{figure}

Los espectros de KOSMOS fueron tomados la misma noche del 4 de diciembre del
2023 en exposiciones de 10 minutos consecutivas. Sin embargo, ambos espectros
muestran una baja razón de señal a ruido (SNR), probablemente por el bajo tiempo
de exposición. Para aumentar el SNR se generó un espectro promedio entre los dos
espectros individuales. Utilizando la función \code{snr\_derived} del paquete
\code{specutils}\footnote{\url{https://specutils.readthedocs.io/en/stable/index.html}}
es posible estimar el SNR basado en solo el espectro medido del archivo. La
función \code{snr\_derived} implementa un algoritmo general para esta tarea,
donde se considera que la señal principal cae en un continuo; el ruido se mide
por la dispersión alrededor del medio del espectro. El espectro promedio se
puede ver en la \reffigure{figuraEspectroApoPromedio}, para el cual el SNR
calculado es de 50.94, a comparación de 39.54 y 38.98 respectivamente de los
espectros individuales en la \reffigure{figuraEspectrosApo}.

\begin{figure}[!ht]
    \centering
    \includegraphics[scale=0.4]{Conclusion/Figures/Figura APO Spectrum Average.png}
    \caption{Espectro promedio de \atoObjIdNoSpace, utilizando el flujo promedio en cada longitud de onda de ambos espectros en la \reffigure{figuraEspectrosApo}.}
    \label{figuraEspectroApoPromedio}
\end{figure}

\subsection{Gaia DR3}

Junto a las curvas de luz fotométricas descritas en la
\refthesissection{muestra:gaia} Gaia DR3 ofrece espectros de aproximadamente 220
millones de las fuentes observadas. Gaia cuenta con un arreglo de
espectro-fotómetros BP y RP (los cuales corresponden a las pasabandas
\textit{Gaia:BP} y \textit{Gaia:RP} respectivamente) que cubren los rangos $[330
- 680] \ \mathrm{nm}$ para BP y $[640 - 1050] \ \mathrm{nm}$ para RP
[\citeyearparen{carrasco_internal_calibration_gdr3_bprp_low_resolution_spectra_2021}].
Cada transito de un objeto por el campo de visión de Gaia contribuye al espectro
promedio de baja resolución; la resolución espectral varía con la longitud de
onda observada, yendo de 100 a 30 para el rango espectral de BP y de 100 a 70
para RP (visto en la figura 3 de
\citeyearparen{carrasco_internal_calibration_gdr3_bprp_low_resolution_spectra_2021}).
Es posible determinar si una fuente en Gaia DR3 cuenta con un espectro por medio
del campo \code{has\_xp\_continuous}. Integrando el espectro en cada rango de
longitud de onda se obtiene el flujo total en Gaia:BP y Gaia:RP.

Para obtener datos limpios de calidad adecuada se someten a un proceso de
calibración extenso, incluyendo la distorsión por la geometría del CCD y la
caracterización del espacio local del objeto\textemdash una revisión extensa del
procesamiento y validación es dada por
\citeyearparen{de_angeli_gdr3_processing_and_validation_bprp_spectra_2023}. El
espectro observado $h_{s,k}(u_i)$ dado un sistema de pseudo-longitudes de onda
$u$ en el catálogo Gaia DR3 se da como una combinación lineal de una función de
base $\varphi$ para una fuente $s$:

\begin{eqfloat}[!ht]
    \begin{equation}
        h_{s,k}(u_i) = \sum_{n=0}^{N - 1} b_{s,n} \sum_{j = -J}^{J} A_k(u_i, u_{i+j}) \varphi_n(u_i+j)
    \end{equation}
\end{eqfloat}

Donde $k$ denota una unidad de calibración (un intervalo de parámetros continuos
cuya variación es baja
[\citeyearparen{carrasco_internal_calibration_gdr3_bprp_low_resolution_spectra_2021}]),
$b_s$ son los coeficientes que contienen la información necesaria para construir
el espectro BP/RP de la fuente, y $A_k$ es el modelo del instrumento. La función
de base elegida por su ortogonalidad, su convergencia a 0 dado un valor de
entrada suficientemente alto, y su centro en $\theta = 0$ se utilizaron
funciones Hermite como las bases del espectro continuo:

\begin{eqfloat}
    \centering
    \begin{equation}
        \begin{split}
            & \varphi_0(\theta) = \pi^{-\frac{1}{4}} e^{-\frac{\theta^2}{2}} \\
            & \varphi_1(\theta) = \sqrt{2} \pi^{-\frac{1}{4}} \theta e^{-\frac{\theta^2}{2}} \\
            & \varphi_n(\theta) = \sqrt{\frac{2}{n}} \theta \varphi_{n-1}(\theta) - \sqrt{\frac{n - 1}{n}} \varphi_{n-2}(\theta)
        \end{split}
    \end{equation}
\end{eqfloat}

Donde se define la transformación lineal de las pseudo-longitudes de onda
$\theta = \Theta \cdot u + \Delta \theta$ con el factor de escala $\Theta$ y un
desplazamiento de $\Delta \theta$ basado en el espectro de la fuente $s$. El
archivo disponible a través del servicio DataLink de Gaia DR3 contiene los
coeficientes de las funciones de base para los espectros BP/RP de
\atoObjIdNoSpace, incluyendo los coeficientes de los errores y la matriz de
correlación entre los coeficientes de las funciones de base. Para facilitar la
lectura y el análisis de este espectro el equipo de Gaia ofrece la herramienta
GaiaXPy\footnote{\url{https://gaiaxpy.readthedocs.io/en/latest/index.html}} que
acepta como entrada el archivo del espectro continuo. Utilizando la función
\code{calibrate} de GaiaXPy este se puede muestrear dado una malla de longitudes
de onda reales en unidades $[\mathrm{nm}]$, dando como resultado el espectro
visto en la \reffigure{figuraEspectroGdr3}.

\begin{figure}[!ht]
    \centering
    \includegraphics[scale=0.45]{Conclusion/Figures/Figura Gaia DR3 Espectro.png}
    \caption{Espectro continuo de \atoObjId obtenido de Gaia DR3 muestreado
    utilizando una malla de longitudes de onda equidistantes en un espacio
    logarítmico para mejorar la resolución del muestreo en el rango más azul.}
    \label{figuraEspectroGdr3}
\end{figure}

\subsection{Análisis: PyHammer}

Para llevar a cabo un análisis rápido de ambos espectros obtenidos se utilizó la
herramienta de clasificación espectral \textbf{PyHammer}\footnote{Disponible en
GitHub: \url{https://github.com/BU-hammerTeam/PyHammer}}. PyHammer\textemdash
descrito por
\citeyearparen{kesseli_pyhammer1_empirical_template_stellar_spectra_classification_2017}
y después por
\citeyearparen{roulston_pyhammer2_classifying_stars_binaries_stellar_templates_2020}
para la versión 2\textemdash fue desarrollado para facilitar la clasificación
automática y manual de espectros estelares de sistemas binarios por medio de
plantillas espectrales para varios tipos de estrellas, desde tipo O hasta tipo L
generados partiendo de datos del catálogo SDSS BOSS (SDSS Baryon Oscillation
Spectroscopic Survey). Aparte del tipo espectral, PyHammer cuenta con plantillas
para determinar la metalicidad y gravedad superficial (la cual se utiliza para
distinguir entre estrellas enanas y gigantes) basado en líneas espectrales de
referencia. 

Utilizamos ambos espectros especificados en esta sección (el espectro promedio
de APO y el espectro muestreado de Gaia DR3 en una malla de longitudes de onda
uniforme en una escala no logarítmica) para corroborar los parámetros obtenidos
del modelo de PHOEBE. Los resultados de ambos análisis de PyHammer se pueden ver
en la \reffigure{figuraAjustePyHammer}. Ambos resultados indican que el sistema
está compuesto de estrellas tipo K, con una metalicidad de $-0.5 \mathrm{dex}$. 

\begin{figure}[!ht]
    \centering
    \includegraphics[scale=0.43]{Conclusion/Figures/Figura PyHammer APO.png} \\
    \vspace{0.6em}
    \includegraphics[scale=0.43]{Conclusion/Figures/Figura PyHammer GDR3.png}
    \caption{Resultados del análisis usando PyHammer para el espectro de APO y
    de Gaia DR3 en la gráfica superior e inferior, respectivamente. Partiendo de
    una estimación inicial automática por PyHammer se ajustó el tipo espectral y
    la metalicidad hasta llegar al mejor ajuste visto en esta figura, utilizando
    el $\chi^2$ reportado en la aplicación como guía.}
    \label{figuraAjustePyHammer}
\end{figure}

Los resultados del análisis de PyHammer coinciden con la temperatura efectiva
derivada usando PHOEBE y las curvas fotométricas de ZTF, dado el rango de
temperaturas efectivas de estrellas tipo K de aproximadamente $3900 - 5300 \
\mathrm{K}$. Sin embargo, mucho cuidado es necesario al interpretar estos
resultados. El espectro de Gaia, a pesar de ser el más completo, es de muy baja
resolución espectral, lo cual causa la perdida de información de las líneas de
emisión y absorción que permitirían una clasificación adecuada del sistema. El
espectro de Gaia también muestra errores significativos en las longitudes de
ondas más cortas, lo cual causa una discrepancia contra el espectro de
plantilla. Al mismo tiempo, el espectro de APO carece de una buena razón de
señal a ruido; a pesar de que la plantilla K4 sea la que mejor se ajusta a los
datos, se puede ver en la gráfica superior de la
\reffigure{figuraAjustePyHammer} un decaimiento repentino en las longitudes de
onda más largas del espectro medido, algo que no se observa en el de Gaia y que
no tenemos una explicación satisfactoria en este trabajo. Dado que la
temperatura efectiva de la componente primaria no tiene un efecto significativo
en la morfología de la curva de luz de un sistema binario en contacto
[\citeyearparen{wadhwa_effective_temperature_light_curve_solutions_cbs_2023}] no
es necesario descartar todo el modelo de PHOEBE para \atoObjIdNoSpace. Un
análisis a mayor profundidad con datos espectroscópicos de mayor precisión
ayudaría a constreñir parámetros adicionales del modelo. En el caso de tener
varios espectros del sistema a lo largo del tiempo se podría construir una curva
de velocidades radiales, la cual se podría ingresar al modelo en PHOEBE
directamente.

% TODO: formatting new page
\newpage
\clearpage

\section{Correlaciones Entre Parámetros}

Una de las mayores dificultades en el ajuste de un modelo de un sistema binario
es debido a las correlaciones entre distintos parámetros del modelo. Algunas
correlaciones son esperadas desde un principio; por ejemplo, en un sistema
binario en contacto existe una correlación fundamental entre la razón de masa
$q$ y el factor de escala de la luminosidad en una pasabanda. Un pequeño ejemplo
se puede ver en la \reffigure{figuraRazonMasaLuminosidadSintetico}; esta
relación se manifiesta en una correlación aproximadamente lineal como se puede
ver en la \reffigure{figuraQ_LuminosidadPdfCorrelacion}.

\begin{figure}[!ht]
    \centering
    \includegraphics[scale=0.37]{Conclusion/Figures/Figura q-Luminosidad Relacion.png}
    \caption{Modelo sintético en el pasabanda Johnson:V de un sistema binario en
    contacto en PHOEBE, manteniendo fijo el factor de relleno en $f = 0.3$. Se
    puede observar como el flujo sintético calculado va aumentando a pesar de
    mantener fijo el factor de escala de la luminosidad en la
    pasabanda\textemdash la cual corresponde al parámetro
    \code{pblum}\textemdash debido en parte al constreñimiento de los radios
    estelares a la razón de masa.}
    \label{figuraRazonMasaLuminosidadSintetico}
\end{figure}

\begin{figure}[!ht]
    \centering
    \includegraphics[scale=0.8]{Conclusion/Figures/Figura q-Luminosidad PDF Correlacion.png}
    \caption{Gráfica de correlación entre la razón de masa $q$ y el factor de
    escala de luminosidad en el pasabanda ZTF:g \code{pblum@lcZtfG}.}
    \label{figuraQ_LuminosidadPdfCorrelacion}
\end{figure}

Esta correlación presenta un gran problema para la solución fotométrica del
sistema: los valores presentados en la
\refthesissection{metodologia:modelocomputacional:mcmc:resultados} no
representan una solución única, por la cual cualquier combinación de valores de
la razón de masa y \code{pblum@lcZtfG} que caigan en esta línea pueden llegar a
ajustar el modelo. Esta dependencia se podría eliminar por completo haciendo uso
de flujos absolutos (por ejemplo, el trabajo hecho por
\citeyearparen{odesse_using_computational_models_params_kepler_binaries_2022}),
lo cual por definición dejaría fijo el factor de escala del flujo sintético en
1. Al mismo tiempo esto haría que el modelo sintético sea sensible a cualquier
parámetro que afecte la luminosidad absoluta del sistema, como las masas de las
componentes y temperaturas efectivas, el cual permitiría muestrear estos
parámetros y obtener distribuciones de densidad de probabilidad de ellos.

Para cuantificar las correlaciones entre parámetros se ajustó una relación
polinomio entre los parámetros muestreados. Este proceso se ejecutó para los
pares de parámetros que mostraron una correlación obvia después de una
inspección visual. Se intentó ajustar una función de grado 1 a 10, de los cuales
se hizo un análisis de convergencia para determinar el menor grado del polinomio
que mejor se ajusta a las muestras. La \reffigure{figuraCorrelacion_q_incl}
muestra la correlación entre la razón de masa y la inclinación orbital del
sistema muestreado; las demás correlaciones se pueden ver en el apéndice en la
\refthesissection{apendice:modelo_computacional_graficas:correlaciones_parametros_muestreados}.

\begin{figure}[!ht]
    \centering
    \includegraphics[scale=0.4]{Conclusion/Figures/Figura q-incl Correlacion Convergencia.png}
    \includegraphics[scale=0.4]{Conclusion/Figures/Figura q-incl Correlacion.png}
    \caption{Función polinomio ajustado a las muestras de la razón de masa
    \code{q} y la inclinación orbital \code{incl@binary}, utilizando los valores
    de la inclinación como el parámetro independiente\textemdash denotado como
    $x$ de la línea solida roja\textemdash para ser consistente con las
    relaciones que aparecen en la \reffigure{figuraMcmcZtfResultadosPrimarios}.
    El ajuste del modelo llega a converger utilizando un polinomio de grado 3 de
    acuerdo a la gráfica superior, el cual es reportado en la gráfica inferior.}
    \label{figuraCorrelacion_q_incl}
\end{figure}

Los ajustes fueron hechos haciendo uso del módulo de regresión lineal de
\href{https://scikit-learn.org/stable/modules/generated/sklearn.linear_model.LinearRegression.html}{\code{scikit-learn}}.
Es importante notar que estas correlaciones aparecen debido a los datos que se
usaron para alimentar el modelo; no es un fenómeno físico presente en el sistema
binario. Cada correlación entre distintos parámetros indica la existencia de una
solución fotométrica no única, dado que cualquier combinación de parámetros que
cumplan con la función de correlación ajustada sería igual de correcta, salvo si
se introducen nuevos datos que permitan restringir aún más los parámetros del
sistema.

% TODO: incluye matriz de covarianza

\section{Extinción Interestelar}

Como se mencionó en la
\refthesissection{metodologia:modelocomputacional:ajuste_luminosidad_teff}, se
utilizó la fotometría de ZTF para determinar la temperatura efectiva del sistema
binario, específicamente los flujos normalizados tal que la información del
color del sistema no se pierda en la conversión de magnitudes. Sin embargo,
cuando se realizó este mismo proceso utilizando los flujos de Gaia DR3 (dados en
cuentas de electrones por segundo), se obtuvo una temperatura efectiva
inconsistente con el resultado de ZTF. Para intentar explicar esta discrepancia
investigamos el efecto de la extinción interestelar debido al polvo en la línea
de visión de \atoObjIdNoSpace.

Para analizar el efecto del polvo en las observaciones fotométricas de Gaia se
utilizó el paquete de Python
\code{dustmaps}\footnote{\url{https://github.com/gregreen/dustmaps}}, el cual
contiene información de varios mapas de polvo hechos en la literatura. Este
paquete contiene mapas que toman en cuenta la distancia al objetivo, calculando
la extinción que experimentaría la luz al pasar por el polvo presente en la
línea de visión. Para este breve análisis se usó el mapa de polvo desarrollado
por \citeyearparen{green_3d_dustmap_gaia_panstarrs_2mass_2019}, el cual utiliza
fotometría de Pan-STARRS 1 y 2MASS en conjunto con las distancias determinadas
utilizando el paralaje reportado en el catálogo de Gaia DR2 para modelar una
distribución tridimensional del polvo interestelar.
\citeyearparen{bailer-jones_estimating_distances_gaia_edr3_2021} corrige errores
sistemáticos en las mediciones del paralaje, reportando distancias corregidas
por medio de métodos probabilísticos, por lo cual decidimos utilizar su
distancia reportada a \atoObjId en vez de una simple inversión del paralaje
reportado por Gaia DR3\footnote{La distancia reportada se encuentra en la tabla
\code{external.gaiaedr3\_distance} en el \textit{Gaia Archive}.}. El mapa de polvo
sintético generado alrededor de \atoObjId se puede ver en la
\reffigure{figuraMapaPolvoLandoltV}.

\begin{figure}[!ht]
    \centering
    \includegraphics[scale=0.54]{Conclusion/Figures/Figura Mapa Polvo Landolt V.png}
    \caption{Distribución del polvo interestelar alrededor de \atoObjId en la
    pasabanda Landolt V, asumiendo $R_V = 3.1$. Este mapa presenta unidades de
    extinción $A_V$, las cuales se obtuvieron utilizando el coeficiente $2.742$
    reportado por
    \citeyearparen{schlafly_measuring_reddening_sdss_calibrating_sfd_2011}. El
    circulo rojo marca la ubicación de \atoObjIdNoSpace.}
    \label{figuraMapaPolvoLandoltV}
\end{figure}

El exceso $E(B-V)$ es un parámetro libre en PHOEBE, el cual se puede utilizar
para parametrizar el enrojecimiento del sistema binario. Al menos que ambas
componentes estelares sean completamente idénticas y experimenten poca
irradiación mutua y distorsión estelar, no es adecuado tratar la extinción
interestelar como un efecto uniforme en toda la curva de luz de un sistema
binario
[\citeyearparen{jones_phoebe_iv_impact_interstellar_extinction_lcs_binaries_2020}].
PHOEBE es capaz de calcular el valor de la extinción observada en cada fase
orbital basado en el perfil espectral del modelo, la función de transmisión de
el pasabanda de la curva de luz, y la ley de enrojecimiento empleada. Sin
embargo, este método tiene la desventaja de no ser compatible con todas las
pasabandas disponibles en PHOEBE, debido a que uno de los requisitos es que
exista una variación particular \code{ext} en los servidores de PHOEBE. Por esta
razón fue necesario tratar las curvas de luz de Gaia por separado de ZTF.

Primero fue necesario ajustar el tiempo de superconjunción del modelo utilizando
las curvas de luz de Gaia. Esto se debe a la presencia de un efecto sistemático
en el tiempo de medición en la fotometría de Gaia; a pesar de que las curvas se
ajusten adecuadamente al mismo periodo orbital, el eclipse principal del modelo
queda desfasado con los datos observados. Tras correr un optimizador de tipo
Nelder-Mead se encuentra el tiempo de superconjunción de los datos de Gaia de
$0.03199 \ \mathrm{d}$ a comparación de $0.02589 \ \mathrm{d}$ para las curvas
de ZTF e Iturbide. Después fue necesario eliminar la restricción del valor
\code{ebv} del modelo por medio de la función \code{flip\_constraint} del
modelo, restringiendo el valor de \code{Av} en su lugar. Esto nos permite
ingresar de manera directa el valor que se obtiene del exceso en el visible de
\code{dustmaps}, utilizando el módulo \code{BayestarQuery} y usando las
coordenadas de \atoObjId junto a su distancia para obtener una muestra del
enrojecimiento en valores de $E(B-V)$. El valor máximo de $0.16$ logró un mejor
ajuste a los datos, como se puede ver en la
\reffigure{figuraPhoebeGaiaExtinguido}.

\begin{figure}[!ht]
    \centering
    \includegraphics[scale=0.45]{Conclusion/Figures/Figura Phoebe Gaia Extinguido.png}
    \caption{Modelo de PHOEBE incorporando la extinción interestelar a los datos
    de Gaia, utilizando un valor de $0.16$ de $E(B-V)$.}
    \label{figuraPhoebeGaiaExtinguido}
\end{figure}

\section{Multi-Modalidad de la PDF Posterior} \label{conclusion:consideraciones_phoebe:multimodalidad_pdf}

Durante el transcurso de este proyecto se llevaron a cabo varias pruebas e
intentos de ajuste del modelo de PHOEBE a los datos observacionales. El primer
ajuste al que se llegó de manera satisfactoria dio como resultado una
combinación distinta de parámetros; el código relevante se encuentra en una
versión previa en el repositorio de GitHub\footnote{El commit con identificador
18d244a4ac8e5f7e17d0a625156af0e9105245f4 muestra la versión del código más
actual con la solución alterna del modelo.}. Las correlaciones entre parámetros
se pueden ver en la figura \reffigure{figuraAltMcmcResultadosZtf}, junto a los
valores e incertidumbres correspondientes en la
\reftable{tablaMcmcResultadosIncertidumbresAlt}. Para este modelo se adoptaron
diferentes parámetros iniciales de los estimadores corridos\textemdash por
ejemplo, el tiempo de superconjunción se dejó fijo en $-0.0375 \ d$, lo cual
causa que la curva de luz en fase esté centrado en el eclipse más profundo,
afectando el parámetro de razón de temperatura\textemdash lo cual al optimizar y
finalmente muestrear llevó el modelo a una región distinta del espacio de
parámetros. Este muestreo se configuró con 160 caminadores en total, igual que
el muestreo descrito en la
\refthesissection{metodologia:modelocomputacional:mcmc}.

\begin{figure}[!ht]
    \centering
    \includegraphics[scale=0.27]{Conclusion/Figures/Figura Alt Solucion MCMC ZTF.png}
    \caption{Gráfica de correlación de una solución alterna que se obtuvo
    utilizando las curvas de ZTF. Incluye el parámetro de molestia
    \code{pblum@lcZtfG}.}
    \label{figuraAltMcmcResultadosZtf}
\end{figure}

{\renewcommand{\arraystretch}{1.5}%
\begin{table}[!ht]
	\centering
	\begin{tabular}{|P{3cm}|P{4cm}|}
		\hline
		\thead{Parámetro}                        & \thead{Valor} \\
		\hline
        $T_{1}$ & $4421.5^{ +6.9 }_{ -7.0 } ~\mathrm{K}$ \\
        \hline
        $T_2 / T_1$ & $0.9621^{ +0.0032 }_{ -0.003 } \mathrm{}$ \\
        \hline
        $f$ & $0.0413^{ +0.0102 }_{ -0.0094 } \mathrm{}$ \\
        \hline
        $i_{\mathrm{orb}}$ & $67.84^{ +0.17 }_{ -0.18 } \mathrm{{}^{\circ}}$ \\
        \hline
        $q$ & $1.52^{ +0.19 }_{ -0.12 } \mathrm{}$ \\
        \hline
        $\mathrm{ Lat }_{\mathrm{spot}}$ & $74.0^{ +9.9 }_{ -10.4 } \mathrm{{}^{\circ}}$ \\
        \hline
        $\mathrm{ Lon }_{\mathrm{spot}}$ & $240.7^{ +2.0 }_{ -2.3 } \mathrm{{}^{\circ}}$ \\
        \hline
        $\mathrm{ Radius }_{\mathrm{spot}}$ & $27.3^{ +5.2 }_{ -3.7 } \mathrm{{}^{\circ}}$ \\
        \hline
        $T_{\mathrm{spot}} / T_2$ & $0.8415^{ +0.0125 }_{ -0.0064 } \mathrm{}$ \\
        \hline
	\end{tabular}
	\caption{Valores e incertidumbres de solución alterna al modelo de PHOEBE.}
	\label{tablaMcmcResultadosIncertidumbresAlt}
\end{table}}

Esta solución fue obtenida después de 6116 iteraciones en total. Sin embargo,
esta solución no es completamente adecuada. El periodo de quemado fue de 4816;
esto se debe a que los caminadores se encontraban atrapados en ciertas regiones
del espacio de parámetros, nunca saliendo por su propia cuenta. La solución para
este problema fue utilizar la capacidad del resolvedor MCMC de PHOEBE para
adoptar el resultado del muestreo como las distribuciones priori de un nuevo
muestreo MCMC. Con el fin de eliminar las muestras de los caminadores inmóviles
se aproximaron las distribuciones posteriores de muestras a distribuciones
continuas Gaussianas multi-dimensionales; a pesar de perder información de la
forma discreta de las muestras\textemdash y al mismo tiempo introduce un sesgo
en la forma inicial de las muestras\textemdash dados suficiente iteraciones los
caminadores tendrán la oportunidad de explorar el espacio fuera de la región
dada por las distribuciones a priori, siguiendo la forma de la distribución
posterior. Debido a que solo se pudieron correr un total de 1345 para la última
ronda de muestreo (el cual incluye 304 iteraciones del periodo de quemado, lo
que eleva el tiempo de quemado total a 5120 iteraciones con solo 996 iteraciones
utilizadas) no es posible decir que este muestreo ha convergido de manera
adecuada.

Debido a la falta de experiencia utilizando PHOEBE, la optimización del tiempo
de cómputo de este muestreo inicial se llevó a cabo de una manera distinta. A
parte del proceso de eliminación de observaciones erróneas que se realizó
(descrito en la
\refthesissubsection{metodologia:modelocomputacional:mcmc:eliminacion_errores}),
se eliminaron la mitad de las observaciones de ambas curvas ZTF:g y ZTF:r,
utilizando los \quotes{slices} de Python. A pesar de que este método redujo el
tiempo de cómputo del modelo hacia adelante, tiene la desventaja de perder mucha
información de observaciones individuales. Al mismo tiempo, esta optimización no
tuvo el mismo efecto significativo que solo computar el modelo hacia adelante
para una malla de fases orbitales determinadas (descrito en la
\refthesissection{metodologia:modelocomputacional:preparacion_modelo}), lo cual
resultó en tiempos promedios de $\sim 3600 \ \mathrm{seg}$ por iteración de MCMC
contra un tiempo promedio de $\sim 500 \ \mathrm{seg}$ por iteración del último
muestreo descrito en la \refthesissection{metodologia:modelocomputacional:mcmc}. 

Este primer intento de explorar el espacio de parámetros se ejecutó en el
servidor \quotes{Alziir} por parte del Instituto de Astronomía Ensenada de la
Universidad Nacional Autónoma de México, en colaboración con el Dr. Raúl Michel
Murillo. Esta computadora cuenta con un procesador AMD Opteron(tm) Processor
6376 de 64 hilos, con un total de 512 GB de memoria RAM; una vez corriendo el
código en el servidor, el proceso llegó a ocupar más de 50 GB de memoria. Debido
al alto tiempo de cómputo para cada iteración, y la solución alterna dada en la
\refthesissubsection{metodologia:modelocomputacional:mcmc:resultados}, esta
solución se abandonó por el momento, pero debería de ser considerada en un
trabajo a futuro en caso que se obtenga una nueva fuente de datos para
constreñir el modelo. 

%%%%%%%%%%%%%%%%%%%%%%%%%%%%%%%%%%%%% Appendix
\appendix
\chapter{Gaia ADQL Encuestas}

\begin{lstlisting}[ language=SQL,
	deletekeywords={IDENTITY},
	deletekeywords={[2]INT},
	morekeywords={clustered, Power},
	framesep=8pt,
	xleftmargin=8pt,
	framexleftmargin=8pt,
	frame=tb,
	framerule=0pt ]
SELECT *,
	sdss_transform.g_sdss - sdss_transform.r_sdss AS g_r_sdss_color,
	sdss_transform.r_sdss - sdss_transform.i_sdss AS r_i_sdss_color
FROM (
	SELECT *,
		-0.13518 + 0.46245 * bp_rp + 0.2517 * Power(bp_rp, 2) - 0.021349 * Power(bp_rp, 3) + phot_g_mean_mag
			AS g_sdss,
		0.29676 - 0.64728 * bp_rp + 0.10174 * Power(bp_rp, 2) + phot_g_mean_mag
			AS i_sdss,
		0.12879 - 0.24662 * bp_rp + 0.027464 * Power(bp_rp, 2) + 0.049465 * Power(bp_rp, 3) + phot_g_mean_mag
			AS r_sdss
	FROM gaiadr2.gaia_source
	WHERE source_id IN (
		SELECT source_id
		FROM user_rcabal01.gaiaonly_distance
	)
) AS sdss_transform
INNER JOIN EXTERNAL.gaiadr2_geometric_distance gdist
	ON sdss_transform.source_id = gdist.source_id 
\end{lstlisting}
\chapter{PHOEBE Gráficas Adicionales} \label{apendice:modelo_computacional_graficas}

\section{Distribuciones Priores Completas} \label{apendice:modelo_computacional_graficas:dist_priores_completas}

\begin{figure}[!ht]
    \centering
    \includegraphics[scale=0.255]{Metodologia/Secciones/ModeloComputacional/Figures/Figura Prior Params Combinadas.png}
    \caption{Distribuciones uniformes de todos los parámetros muestreados
    utilizando cadenas de Monte Carlo (MCMC). La topología compleja que se ve en
    cada cuadrante se debe a las muestras que PHOEBE toma para cada parámetro en
    el proceso de generar la gráfica, no debido a la distribución almacenada por
    PHOEBE. Los priores, junto a esta gráfica, fueron generados en el Notebook
    \href{https://github.com/KnightIV/UANL_MAPTA_Observaciones/blob/main/analisis/phoebe_model/sampling/updated-data-mcmc-sampling.ipynb}{\code{updated-data-mcmc-sampling.ipynb}}.}
    \label{figuraPhoebePrioresCombinadasZtf}
\end{figure}

\section{Distribuciones de Densidad de Probabilidad Completas} \label{apendice:modelo_computacional_graficas:resultados_pdfs_completas}

\begin{figure}[!ht]
    \centering
    \includegraphics[scale=0.3]{Apendice/Figures/Figura MCMC ZTF Resultados Completos.png}
    \caption{Resultados completos del muestreo MCMC utilizando los datos de ZTF.
    Aquí se incluye el parámetro de molestia \code{pblum@primary@lcZtfG}, el
    cual representa la luminosidad en la pasabanda ZTF:g de la componente
    primaria, utilizada como el factor de escala para la curva de luz de ZTF:g y
    la temperatura efectiva del sistema reflejada por la curva de ZTF:r. Se
    puede apreciar la correlación lineal entre la razón de masa $q$ y la
    luminosidad de pasabanda.}
    \label{figuraPhoebeMcmcResultadosCompletos}
\end{figure}
%-------------------------------------REFERENCES
\chapter*{Agradecimientos}
This work has made use of data from the European Space Agency (ESA) mission
\textit{Gaia} (\url{https://www.cosmos.esa.int/gaia}), processed by the Gaia
Data Processing and Analysis Consortium (DPAC,
\url{https://www.cosmos.esa.int/web/gaia/dpac/consortium}). Funding for the DPAC
has been provided by national institutions, in particular the institutions
participating in the {\it Gaia} Multilateral Agreement.

This job has made use of the Python package GaiaXPy, developed and maintained by
members of the Gaia Data Processing and Analysis Consortium
(\href{https://www.cosmos.esa.int/web/gaia/dpac/consortium}{DPAC}), and in
particular, Coordination Unit 5 (CU5), and the Data Processing Centre located at
the Institute of Astronomy, Cambridge, UK (DPCI).

Based on observations obtained with the Samuel Oschin Telescope 48-inch and the 60-inch Telescope at the Palomar
Observatory as part of the Zwicky Transient Facility project. ZTF is supported by the National Science Foundation under Grants
No. AST-1440341 and AST-2034437 and a collaboration including current partners Caltech, IPAC, the Oskar Klein Center at
Stockholm University, the University of Maryland, University of California, Berkeley , the University of Wisconsin at Milwaukee,
University of Warwick, Ruhr University, Cornell University, Northwestern University and Drexel University. Operations are
conducted by COO, IPAC, and UW.

This research made use of Astropy,\footnote{\url{http://www.astropy.org}} a
community-developed core Python package for Astronomy \citeyearparen{astropy}.

This research has made use of the SIMBAD database, operated at CDS, Strasbourg, France.

This research made use of ccdproc, an Astropy package for
image reduction \citeyearparen{ccdproc241}.
\clearpage

\printbibliography

\end{document}
