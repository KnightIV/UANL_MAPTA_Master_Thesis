\chapter{Catálogos}

\section{Gaia} \label{muestra:sec:gaia}

Originalmente denominado como \textit{GAIA}, la misión de \gaia fue lanzada por
la \textbf{Agencia Espacial Europea (ESA)} el 19 de Diciembre del 2013, con el
objetivo de generar un mapa tridimensional de nuestra Galaxia, la Vía Láctea.
Esto incluye calcular las propiedades astrométricas y astrofísicas de sus
fuentes observadas con mayor precisión que cualquier otro catálogo publicado
previamente. Para lograr esto se utiliza un satélite espacial, el cual está
denominado como \gaiaNoSpace, ubicado en el punto de Lagrange L2 con respecto al
sistema Sol-Tierra. Desde este punto la nave tiene una vista sin obstrucciones
que le permite observar una cantidad de estrellas enorme, con \textapproxtilde
1,000 millones de fuentes visibles con los instrumentos del satélite
\gaiaNoSpace. \autocite{gaiaMission}

\subsection{Fotometría}

\subsection{Data Release 2} \label{muestra:sec:gaia:dr2}

Para facilitar el acceso público a los datos recabados por la misión de \gaia la
ESA ha escogido liberar los datos públicamente mediante los van recibiendo y
procesando. Estos son conocidos como los \textbf{Data Releases}. Este trabajo se
basa en el \textbf{Data Release 2}, el cuál de ahora en adelante será denominado
simplemente \textbf{GDR2}. Este catálogo está compuesto de las observaciones
hechas por \gaia entre las fechas de 25 de Julio del 2014 y el 23 de Mayo del
2016, un periodo de tiempo de 22 meses en total. \autocite{gaiaDr2} GDR2 consiste
de \num{1692919135} de fuentes individuales. Existe una gran diversidad de objetos
dentro de este catálogo, desde estrellas de secuencia principal, asteroides
dentro del sistema solar, hasta estrellas variables en las regiones más lejanas
en la Galaxia.

Los datos utilizados en este estudio fueron accedidos a través de el
\textit{Gaia Archive}\footnote{\url{https://gea.esac.esa.int/archive/}}, una
herramienta libre publicada por la ESA. Este cuenta con una interfaz de
ADQL\footnote{\url{https://www.ivoa.net/documents/ADQL/20180112/PR-ADQL-2.1-20180112.html}},
un lenguaje estructurado para hacer consultas a la base de datos de
\gaiaNoSpace, incluyendo tablas conectando los datos de \gaia con otros
catálogos, por ejemplo el catálogo de SDSS.

\section{Sloan Digital Sky Survey}

La colección de catálogos \textbf{Sloan Digital Sky
Survey}\footnote{\url{https://www.sdss.org}} (de ahora en adelante será referido
como \textbf{SDSS}) compila varias fuentes de datos astronómicos y astrofísicos
en un sitio centralizado, con el objetivo de crear un mapa tridimensional del
Universo con una precisión no vista antes. Estos incluyen imágenes de objetos
astronómicos en varios colores, acompañados de un espectro obtenido como parte
de esta misión. Para los finales del siglo XX habían surgido avances
tecnológicos que llegarían a revolucionar la astronomía observacional. De estos,
los de mayor interés ocurrieron con los detectores de estado sólido y en la
capacidad computacional de procesamiento. Partiendo de estos empezaron a
desarrollar la infraestructura necesaria para recabar datos fotométricos y
espectroscópicos.

El instrumento principal utilizado es el telescopio de 2.5m, ubicado en el
observatorio \textit{Apache Point Observatory}, descrito a detalle en
\autocite{sdss2_5mTelescope}. Este telescopio de diseño de Ritchey-Chrétien
alimenta dos instrumentos separados; una CCD multi-banda de ancha área, y un par
de espectrógrafos alimentados por fibra óptica. Su construcción empezó en 1998,
pero no fue hasta el año 2000 que estuvo operacional.

\subsection{Data Release 9}

SDSS libera datos en colecciones iterativas; es decir cada Data Release (DR)
liberado contiene todas las observaciones que forman parte del DR previo,
agregando los datos recabados durante el periodo de observación para el DR
actual. Cada DR cae bajo una fase de operaciones de SDSS, delimitado tanto por
las fechas de observaciones como por los instrumentos y tipos de datos
disponibles. Para el periodo operacional de
\hyperref[muestra:sec:gaia:dr2]{GDR2} el catálogo más actual de SDSS era el DR9
publicado como parte de SDSS-III
\footnote{\url{https://www.sdss3.org/index.php}}. Esta tercera fase fue marcada
por una gran mejora del equipo espectroscópico, instalando nuevos instrumentos
con los cuales pudieron analizar la dinámica de nuestra Galaxia, al igual que
otras galaxias y planetas gaseosos extra-solares. 

