\section{ATO J339.9469+45.1464 - EclBin\_Candidate}

% \textbf{\atoObjId} fue clasificado como candidata a binaria eclipsante en \citet{atlasATOObjectDiscovery}. 

\citet{atlasATOObjectDiscovery} es un estudio donde se realizó una búsqueda de estrellas variables dentro del catálogo del \textbf{Asteroid Terrestrial-impact Last Alert System} (\textbf{ATLAS}), aprovechando su cobertura de aproximadamente \num{13000} $\mathrm{deg}^2$ al menos 4 veces por noche. Esta cadencia de observación es ideal para observar estrellas variables; el tiempo de observación es suficientemente corto para obtener una curva de luz lo suficientemente completa para estudiar estos sistemas. Lograron clasificar las estrellas variable del catálogo en 15 distintas categorías de la morfología de sus curvas de luz; de estas reportan que \num{74700} fuentes son binarias eclipsantes. A pesar de haber confirmado la clasificación de estas fuentes, aún quedan varios sistemas cuya naturaleza es desconocido, sus únicos descriptores vienen siendo una clasificación tentativa.

\textbf{\atoObjId} está clasificado como una de estas candidatas a binaria eclipsante. 