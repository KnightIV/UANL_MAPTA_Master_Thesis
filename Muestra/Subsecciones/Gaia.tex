\section{Gaia} \label{muestra:sec:gaia}

Originalmente denominado como \textit{GAIA}, la misión de \gaia fue lanzada por
la \textbf{Agencia Espacial Europea (ESA)} el 19 de Diciembre del 2013, con el
objetivo de generar un mapa tridimensional de nuestra Galaxia, la Vía Láctea.
Esto incluye calcular las propiedades astrométricas y astrofísicas de sus
fuentes observadas con mayor precisión que cualquier otro catálogo publicado
previamente. Para lograr esto se utiliza un satélite espacial, el cual está
denominado como \gaiaNoSpace, ubicado en el punto Lagrangiano L2 con respecto al
sistema Sol-Tierra. Desde este punto la nave tiene una vista sin obstrucciones
que le permite observar una cantidad de estrellas enorme, con \textapproxtilde
1,000 millones de fuentes visibles con los instrumentos del satélite
\gaiaNoSpace. \citet*{gaiaMission}

\comment{
\subsection{Fotometría}
}

\subsection{Data Release 2} \label{muestra:sec:gaia:dr2}

Para facilitar el acceso público a los datos recabados por la misión de \gaia la
ESA ha escogido liberar los datos públicamente mediante los van recibiendo y
procesando. Estos son conocidos como los \textbf{Data Releases}. Este trabajo se
basa en el \textbf{Data Release 2}, el cuál de ahora en adelante será denominado
simplemente \textbf{GDR2}. Este catálogo está compuesto de las observaciones
hechas por \gaia entre las fechas de 25 de Julio del 2014 y el 23 de Mayo del
2016, un periodo de tiempo de 22 meses en total. \citet*{gaiaDr2} GDR2 consiste
de 1,692,919,135 de fuentes individuales. Existe una gran diversidad de objetos
dentro de este catálogo, desde estrellas de secuencia principal, asteroides
dentro del sistema solar, hasta estrellas variables en las regiones más lejanas
en la Galaxia.

Los datos utilizados en este estudio fueron accedidos a través de el
\textit{Gaia Archive}\footnote{\url{https://gea.esac.esa.int/archive/}}, una
herramienta libre publicada por la ESA. Este cuenta con una interfaz de
ADQL\footnote{\url{https://www.ivoa.net/documents/ADQL/20180112/PR-ADQL-2.1-20180112.html}},
un lenguaje estructurado para hacer consultas a la base de datos de
\gaiaNoSpace, incluyendo tablas conectando los datos de \gaia con otros
catálogos, por ejemplo el catálogo de SDSS.