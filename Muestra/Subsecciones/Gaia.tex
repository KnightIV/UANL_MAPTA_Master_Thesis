\section{Gaia} \label{muestra:sec:gaia}

La misión de \gaia fue lanzada por la \textbf{Agencia Espacial Europea (ESA)} el 19 de Diciembre del 2013, con el objetivo de generar un mapa tridimensional de nuestra Galaxia, la Vía Láctea. Esto incluye calcular las propiedades astrométricas y astrofísicas de sus fuentes observadas con mayor precisión que cualquier otro catálogo publicado previamente. Para lograr esto se utiliza un satélite espacial, el cual está denominado como \gaiaNoSpace, ubicado en el punto Lagrangiano L2 con respecto al sistema Sol-Tierra. Desde este punto la nave tiene una vista sin obstrucciones que le permite observar una cantidad de estrellas enorme, con \textapproxtilde 1,000 millones de fuentes visibles con los instrumentos del satélite \gaiaNoSpace. \citet*{gaiaMission}

\subsection{Data Release 2} \label{muestra:sec:gaia:dr2}
