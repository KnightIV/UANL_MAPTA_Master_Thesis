\subsection{Sloan Digital Sky Survey}

La colección de catálogos \textbf{Sloan Digital Sky
Survey}\footnote{\url{https://www.sdss.org}} (de ahora en adelante será referido
como \textbf{SDSS}) compila varias fuentes de datos astronómicos y astrofísicos
en un sitio centralizado, con el objetivo de crear un mapa tridimensional del
Universo con una precisión no vista antes. Estos incluyen imágenes de objetos
astronómicos en varios colores, acompañados de un espectro obtenido como parte
de esta misión. Para los finales del siglo XX habían surgido avances
tecnológicos que llegarían a revolucionar la astronomía observacional. De estos,
los de mayor interés ocurrieron con los detectores de estado sólido y en la
capacidad computacional de procesamiento. Partiendo de estos empezaron a
desarrollar la infraestructura necesaria para recabar datos fotométricos y
espectroscópicos.

El instrumento principal utilizado es el telescopio de 2.5m, ubicado en el
observatorio \textit{Apache Point Observatory}, descrito a detalle en
\citet*{sdss2_5mTelescope}. Este telescopio de diseño de Ritchey-Chrétien
alimenta dos instrumentos separados; una CCD multi-banda de ancha área, y un par
de espectrógrafos alimentados por fibra óptica. Su construcción empezó en 1998,
pero no fue hasta el año 2000 que estuvo operacional.

\subsubsection{Data Release 9}

SDSS libera datos en colecciones iterativas; es decir cada Data Release (DR)
liberado contiene todas las observaciones que forman parte del DR previo,
agregando los datos recabados durante el periodo de observación para el DR
actual. Cada DR cae bajo una fase de operaciones de SDSS, delimitado tanto por
las fechas de observaciones como por los instrumentos y tipos de datos
disponibles. Para el periodo operacional de
\hyperref[muestra:sec:gaia:dr2]{GDR2} el catálogo más actual de SDSS era el DR9
publicado como parte de SDSS-III
\footnote{\url{https://www.sdss3.org/index.php}}. Esta tercera fase fue marcada
por una gran mejora del equipo espectroscópico, instalando nuevos instrumentos
con los cuales pudieron analizar la dinámica de nuestra Galaxia, al igual que
otras galaxias y planetas gaseosos extra-solares. 

