\chapter{PHOEBE - Modelo Computacional}

Como se ha planteado, los sistemas binarios estelares ofrecen una oportunidad de
analizar el comportamiento y la estructura estelar que sería imposible deducir
en estrellas aisladas. Sin embargo, el análisis analítico es una herramienta
restringida en cuanto a la cantidad de información que puede extraer de una
curva de luz; las ecuaciones que rigen un sistema binario no tienen una solución
analítica. Para resolver este problema, se hacen códigos capaces de simular la
física de un sistema binario, el cual partiendo de ciertos parámetros estelares
y orbitales pueden reproducir los datos observacionales de un sistema
equivalente en la bóveda celeste.

En el campo de sistemas binarios estelares, uno de los primeros códigos con
mayor impacto es el código \textbf{Wilson-Devinney}, descrito por primera vez en
\autocite{wilson_devinney_realization_of_accurate_binary_lcs_wd_1971}. El código
de Wilson-Devinney\textemdash referenciado como el código \textbf{WD} en varias
publicaciones\textemdash parte del modelo de Roche para representar las
superficies de las estrellas, lo cual le permite hacer un trato adecuado de
fenómenos físicos importantes como el oscurecimiento al limbo y el
oscurecimiento gravitacional. A pesar de sobrepasar 5 décadas de edad el código
WD sigue en uso actualmente en proyectos de investigación (por ejemplo,
\autocite{li_extremely_low_mass_ratio_wd_analysis_2022}), de los cuales se
obtienen los parámetros físicos de un sistema binario estelar observado. Este
trabajo de tesis de maestría utilizó la librería \textbf{PHOEBE}
(\textbf{PH}ysics \textbf{O}f \textbf{E}clipsing \textbf{B}inari\textbf{E}s),
basado en el código WD.

\section{\quotes{Modelo Hacia Adelante}}

El propósito principal de códigos como PHOEBE cae en su capacidad para generar
un \quotes{\textbf{modelo hacia adelante}} (traducido de forma directa de su nombre
en inglés: \textbf{forward model}). Se parte de un modelo del sistema\textemdash
en el caso de un sistema binario, este viene siendo el modelo de Roche junto a
una formulación de las superficies estelares\textemdash el cual se va integrando
en el tiempo, produciendo como resultado datos sintéticos observables como una
curva de luz fotométrica o una curva de velocidades radiales. Un ejemplo de un
sistema \quotes{de juguete} se puede ver en la
\reffigure{figuraPhoebeObservablesSinteticos}. Es importante notar que estos
modelos se trabajan en el espacio fase de la órbita de un sistema; en casos
donde el sistema no experimente algún cambio significativo a lo largo del
tiempo, es suficiente computar el modelo para cada fase orbital observada, dado
que una campaña de observación adecuada abarcaría las mismas fases orbitales más
de una vez.

\begin{figure}[!ht]
	\centering
	\xincludegraphics[scale=0.253, label=\textbf{a)}, labelbox=true, pos=nw, fontsize=\Large]{Introduccion/Figures/Figura PHOEBE LC Sintetico.png}
	\xincludegraphics[scale=0.253, label=\textbf{b)}, labelbox=true, pos=nw, fontsize=\Large]{Introduccion/Figures/Figura PHOEBE RVs Sintetico.png}
	\caption{Datos sintéticos generados usando PHOEBE. Estas curvas representan
	un sistema binario separado, donde $M_1 = 1.0 \ \mathrm{M}_{\odot}$, $M_2 =
	0.5 \ \mathrm{M}_{\odot}$, $R_1 = 2.0 \ \mathrm{R}_{\odot}$, $R_2 = 1.2 \
	\mathrm{R}_{\odot}$, $T_1 = T_2 = 6000 \ \mathrm{K}$, inclinación
	$i_{\mathrm{orb}} = 90^{\circ}$ y periodo orbital $P_{orb} = 1.0 \
	\mathrm{d}$ en una órbita sincrónica. La figura muestra dos diferentes tipos
	de observables: \textbf{a)} Curva de luz fotométrica, cuyas variaciones se
	deben a los eclipses en el sistema. \textbf{b)} Curva de velocidades
	radiales de ambas componentes, donde el movimiento de las estrellas
	individuales a lo largo de nuestra línea de visión causa fluctuaciones se
	observa debido al efecto de Doppler.} 
	\label{figuraPhoebeObservablesSinteticos}
\end{figure}

Para generar datos sintéticos observables de un sistema binario, es necesario
que PHOEBE tome ciertos aspectos en consideración, descritos a continuación en
este capitulo.

\subsection{Discretización de la Superficie Estelar}

Partiendo del modelo de Roche es como se determina la superficie de ambas
componentes, siguiendo el principio de superficies equipotenciales. Sin embargo,
una descripción analítica de las variaciones de las propiedades estelares
resultaría en una complejidad de tiempo intratable del problema; a pesar de
permitirnos modelar pequeñas variaciones en los valores de cada parámetro
superficial, este no es una opción realista dado la capacidad de computo actual.
Es por esto que PHOEBE implementa un método el cual aproxima la superficie de
las estrellas a un muestreo uniforme de puntos determinados por el modelo de
Roche. Sin embargo, se requiere un tratamiento adicional del modelo de Roche; la
\refequation{ecuacionRocheExcentricaAsincronica} viene definida en coordenadas
esféricas, lo cual causaría una distribución no uniforme en el tamaño de los
elementos superficiales no deseada (aparte de causar un tipo de \quotes{costura}
a lo largo del ecuador de la estrella \citetbookchapter{phoebeScientificReference}{5.1.1}).

Para resolver este problema, se transforma la
\refequation{ecuacionRocheExcentricaAsincronica} a un sistema de coordenadas
cilíndrico de acuerdo a las siguientes transformaciones:

\begin{eqfloat}[!ht]
	\centering
	\begin{equation}
		\begin{split}
			& x = \varrho_{\perp} \cos{\phi} \\
			& y = \varrho_{\perp} \sin{\phi} \\
			& z = z
		\end{split}
	\end{equation}
\end{eqfloat}

Donde $\phi$ representa la longitud, $\varrho_{\perp}$ es la componente en el
plano orbital de la distancia al elemento superficial, y $z$ mantiene su
definición del sistema de coordenadas cartesianas. Utilizando estas
transformaciones se llega a una expresión del potencial de Roche en coordenadas
cilíndricas:

% TODO: new page to group equation and text together
\newpage

\begin{eqfloat}[!ht]
	\centering
	\begin{equation}
		\Omega = \frac{1}{\sqrt{\varrho_{\perp} + z^2}} + q \left(\frac{1}{\sqrt{\delta^2 + \varrho_{\perp}^2 + z^2 - 2 \varrho_{\perp} \delta \cos{\phi}}} - \frac{\varrho_{\perp} \cos{\phi}}{\delta^2} \right) + \frac{F^2 \left(1 + q\right) \varrho_{\perp}^2}{2}
	\end{equation}
	\blankcaption
	\label{ecuacionRocheCilindrica}
\end{eqfloat}

Parecido a la \refequation{ecuacionRadioPolarEstelarEquivalencia} se puede
llegar a una expresión similar para encontrar el valor de $\varrho_{\perp}$ que
le corresponde a los valores dados de $\phi$ y $z$, utilizando un método para
encontrar raíces como el de \textit{Newton-Raphson}:

\begin{eqfloat}[!ht]
	\centering
	\begin{equation}
		\begin{split}
			f(\varrho_{\perp}) = \eqnmark[MyDarkRed]{omega}{\frac{1}{\sqrt{\varrho_{\perp} + z^2}} + q \left(\frac{1}{\sqrt{\delta^2 + \varrho_{\perp}^2 + z^2 - 2 \varrho_{\perp} \delta \cos{\phi}}} - \frac{\varrho_{\perp} \cos{\phi}}{\delta^2} \right) + \frac{F^2 \left(1 + q\right) \varrho_{\perp}^2}{2}} \\
			\eqnmark[MyDarkBlue]{omegaPol}{- \frac{1}{z_{\textrm{pol}}} - \frac{q}{\sqrt{z_{\textrm{pol}}^2 + \delta^2}}}
		\end{split}
	\end{equation}
	\annotate[yshift=-0.2em]{below, left}{omega}{Potencial de Roche $\Omega$}
	\annotate[yshift=-0.2em]{below, left}{omegaPol}{Potencial de referencia en el polo $\Omega_{\mathrm{pol}}$}
	\blankcaption
	\vspace{0.4em}
	\label{ecuacionRadioCilindrica}
\end{eqfloat}

Donde se define $z_{\mathrm{pol}} \equiv \varrho_{\mathrm{pol}}$ como el valor
de referencia del radio al polo de la estrella
\citetbookchapter{phoebeScientificReference}{5.1.1}. Al hacer uso de la derivada
$f\prime (\varrho_{\perp})$\textemdash la cual se puede obtener como lo muestra
\citetbookchapter{phoebeScientificReference}{5.1.1}\textemdash es posible
implementar un método para encontrar raíces del sistema. 