\chapter{PHOEBE}

Como se ha planteado, los sistemas binarios estelares ofrecen una oportunidad de
analizar el comportamiento y la estructura estelar que sería imposible deducir
en estrellas aisladas. Sin embargo, el análisis analítico es una herramienta
restringida en cuanto a la cantidad de información que puede extraer de una
curva de luz; las ecuaciones que rigen un sistema binario no tienen una solución
analítica. Para resolver este problema, se hacen códigos capaces de simular la
física de un sistema binario, el cual partiendo de ciertos parámetros estelares
y orbitales pueden reproducir los datos observacionales de un sistema
equivalente en la bóveda celeste.

En el campo de sistemas binarios estelares, uno de los primeros códigos con
mayor impacto es el código \textbf{Wilson-Devinney}, descrito por primera vez en
\autocite{wilson_devinney_realization_of_accurate_binary_lcs_wd_1971}. El código
de Wilson-Devinney\textemdash referenciado como el código \textbf{WD} en varias
publicaciones\textemdash parte del modelo de Roche para representar las
superficies de las estrellas, lo cual le permite hacer un trato adecuado de
fenómenos físicos importantes como el oscurecimiento al limbo y el
oscurecimiento gravitacional. A pesar de sobrepasar 5 décadas de edad el código
WD sigue en uso actualmente en proyectos de investigación (por ejemplo,
\autocite{li_extremely_low_mass_ratio_wd_analysis_2022}), de los cuales se
obtienen los parámetros físicos de un sistema binario estelar observado. Este
trabajo de tesis de maestría utilizó la librería \textbf{PHOEBE}
(\textbf{PH}ysics \textbf{O}f \textbf{E}clipsing \textbf{B}inari\textbf{E}s),
basado en el código WD.

\section{\quotes{Modelo Hacia Adelante}}

