\section{Sistemas Binarios}

La gran mayoría de sistemas estelares dentro de nuestra Galaxia no son aquellos
solitarios como nuestro propio sistema solar, si no que son compuestas de dos o
más estrellas ubicadas en corta aproximación de una a otra, a ordenes de
unidades astronómicas (AU por sus siglas en inglés). Estos sistemas multiples se
pueden clasificar con mayor precisión para aquellos compuestos de solo dos
estrellas, denominados como \textit{sistemas binarios}. Dentro de un sistema
binario la corta separación orbital entre ambas estrellas da como consecuencia a
fenómenos que surgen mediante la interacción entre las componentes, tanto como
la interacción gravitacional debido a sus masas, como a la física interesante
que ocurre en el caso de interacciones de material entre una estrella a otra. 

\subsection{Clasificaciones Observacionales}

Dependiendo del método de detección y las propiedades aparentes del sistema se
puede clasificar un sistema binario de estrellas. Estas clasificaciones son
independiente de sus propiedades físicas, como la clase espectral de cada
estrella o sus masas individuales. Al determinar su clasificación observacional
se puede delimitar las técnicas observacionales que son viables para recabar
datos del sistema; un sistema astrométrico sería indistinguible de uno
espectroscópico si uno intenta identificar las componentes individuales a simple
vista, o con un telescopio demasiado débil para el trabajo.

Las \textbf{binarias visuales} son aquellos cuya separación orbital aparente es
suficientemente grande para distinguir las dos estrellas individuales en la
bóveda celeste. A pesar de que se puede trazar la órbita de la secundaria con
varios años de observaciones, se requiere de cálculos adicionales para
determinar la órbita exacta de las componentes. Esto se debe a la inclinación
del sistema con respecto al eje de observación hacia la Tierra; solo es posible
observar \quotes{una proyección del elipse orbital relativo en el plano del
cielo,} aunque esto se puede superar usando el hecho de que la estrella primaria
aparentemente inmóvil debe de estar presente \quotes{en un punto focal de la
órbita relativa.} \citetbookchapter{fundamentalAstronomy}{10}

Las \textbf{binarias espectroscópicas} presentan variaciones periódicas en sus
espectros, en donde las líneas espectrales detectadas \quotes{oscilan
periodicamente alrededor de la longitud de onda promedio}
\citetbookchapter{astronomyPhysicalPerspective}{5}. Esto se observa
debido al \textit{desplazamiento de Doppler}, lo cual causa que la frecuencia de
un fotón se recorra hacia frecuencias más pequeñas (azules) o más grandes
(rojas) dependiendo de su velocidad radial con respecto al observador, si se va
acercando o alejando, respectivamente. Estas también pueden identificadas al
observar dos distintos grupos de líneas espectrales, el cual es resultado de la
contribución de ambas estrellas.

Las \textbf{binarias astrométricas}, al igual que las espectroscópicas, solo
muestran una componente visible al ser observada, al contrario de las binarias
visibles. Sin embargo, una binaria astrométrica difiera de las otras dos
categorías definidas en cuestión de su movimiento observado en la bóveda
celeste. Estas muestran un movimiento errático y no-lineal, algo que no se
esperaría ver en una estrella solitaria dado su inercia según la primera ley de
Newton. Estas perturbaciones son causadas por una estrella secundaria no
aparente al observar el sistema. 

\subsubsection{Binarias Eclipsantes}

Una de las propiedades más importantes de identificar de un sistema binario es la \textit{inclinación} de su órbita con respecto a nuestra línea de visión desde el sitio de observación (ya sea la Tierra en caso de un observatorio terrestre o un punto lejano dentro del sistema solar para un telescopio espacial). En dado caso que un sistema tenga una inclinación 

\subsection{Binarias en Contacto}