\section{Sistemas Binarios}

A diferencia de nuestro sistema solar, una gran mayoría de sistemas dentro de
nuestra Galaxia están compuestas de multiples estrellas a tan solo unos cuantos
radios solares de separación. Una parte de estos son los \textit{sistemas
binarios}: aquellos compuestos de dos distintas estrellas. En este tipo de
sistemas la distancia entre las estrellas individuales cae dentro de ordenes de
unas cuantas unidades astronómicas, ordenes de magnitudes más pequeñas que la
distancia desde el Sol hacia la estrella más cercana a nuestro sistema solar,
Proxima Centauri, a cuatro años luz. 

\comment{
Estas estrellas van
evolucionando juntas desde su formación dentro de una nube molecular común,
ligadas una con otra en su camino evolutivo. 
}

\comment{
Durante sus primeros momentos de vida se logran
formar una gran cantidad de estas estrellas; sin embargo la mayoría de las
estrellas formadas durante esta etapa de tiempo no llegan a la edad vieja,
siendo consumidas por estrellas más masivas dentro de la nube molecular. 
}