\section{Sistemas Binarios}

A diferencia de nuestro sistema solar, una VC está compuesta de dos distintas
estrellas: una estrella enana blanca y una estrella de secuencia principal.
Dentro de un sistema binario ambas estrellas existen en proximidad de una a otra
en ordenes de unos cuantos UA, comparado con las distancias entre distintos
sistemas solares cuya magnitud recae en el orden de años luz. Estas estrellas
van evolucionando juntas desde su formación dentro de una nube molecular común,
donde las estrellas nacen. Durante sus primeros momentos de vida se logran
formar una gran cantidad de estas estrellas; sin embargo la mayoría de las
estrellas formadas durante esta etapa de tiempo no llegan a la edad vieja,
siendo consumidas por estrellas más masivas dentro de la nube molecular. 