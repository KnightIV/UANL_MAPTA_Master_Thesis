\section{Estrellas}

Las \textit{estrellas} son de los objetos más fundamentales e importantes en el
estudio de los astros. \citet*{anIntroStellarAstro::chapter1_basicConcepts}
define una estrella como \quotes{un objeto celeste en el cual existe, o alguna
vez existió (en el caso de una estrella muerta) fusión de hidrógeno sostenido en
su núcleo.} Dentro del núcleo de cada estrella activa se forjan nuevos elementos
más pesados debido a estas reacciones termonucleares, primero generando el helio
mediante el proceso de $4 \ce{^1H} \rightarrow \ce{^4He} + E$, hasta llegar a la
fusion del helio para formar el carbono mediante el \textit{proceso
triple-alfa}.

A pesar de este requisito presentado para clasificar a un objeto como una
estrella existen varios tipos de estrellas en nuestra Galaxia. Entre ellas
muestran una gran variedad de propiedades físicas, tanto como su brillo y color,
como su camino evolutivo a lo largo del tiempo, en donde sus comportamientos
divergen en lo que van agotando su combustible.

