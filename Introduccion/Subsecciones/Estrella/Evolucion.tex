\subsection{Evolución}

En el transcurso del tiempo una estrella va a estar sujeta a ciertos cambios en
su estructura característica. Esto se debe a que podemos considerar a una
estrella cómo un objeto aislado en el espacio, lo cual significa que no tendrá
algún ingreso de material significativo para reemplazar el combustible
\quotes{quemado} en las reacciones termonucleares. A lo largo del tiempo la
composición física y química de la estrella deberán cambiar para mantener el
equilibrio termodinámico.

El combustible primario de una estrella viene siendo el hidrógeno atómico, el
cual se fusiona con otros átomos (protones individuales) libres, resultando en
la producción de grandes cantidades de energía en forma de radiación, dejando en
el lugar de los dos protones un átomo de helio. El helio no es inmediatamente
util para la estrella cómo combustible; el helio requiere temperaturas más altas
de las que actualmente están presentes en el núcleo para poder fusionar entre
si. Todas las estrellas conocidas pasan por esta etapa de evolución estelar;
mientras que una estrella dependa principalmente del hidrógeno para brillar se
dice que está en su etapa de \textit{secuencia principal}.