\subsection{Formación}

% TODO: add citations from "Formación estelar" presentation I did in MHD course
% Agrega paginas web

El \textbf{medio interestelar} (ISM por sus siglas en inglés) es el hogar de
toda la materia que existe dentro de nuestra Galaxia. Está compuesta de todo el
polvo, gas, e incluso los rayos cósmicos que atraviesan el espacio hasta llegar
a la Tierra donde los podemos detectar. Mucho de este material termina siendo
acumulado en volúmenes discretos en el espacio. Estas nebulosas llegan a
temperaturas extremadamente frías, llegando hasta los 10-20° K en su estado de
equilibrio. En cuanto esta nube es perturbada por alguna fuerza externa su
configuración empieza a evolucionar. Empiezan a aparecer regiones de mayor
densidad, donde el material se comienza a acumular, aumentando la masa del
volumen local y junto a ella la fuerza gravitacional que ejerce en el resto de
la nube. A lo largo del tiempo el material se empieza a calentar por el mismo
colapso gravitacional, en la cual la energía gravitacional es convertida a
energía térmica en las partículas. En cuanto el material logre calentarse lo
suficiente para que el hidrógeno puede empezar a fusionar a helio el conjunto de
material se convierte en una \textbf{protoestrella}, la primera fase en la
formación y evolución de una estrella.

% TODO: probably add symbols table in thesis for M_\odot (and probably others to
% be used in rest of doc)

El proceso de colapso gravitacional necesariamente requiere una masa mínima para
que las reacciones termonucleares ocurran de una manera sustentable tal como se
observa en una estrella. El material acumulado debe consistir de al menos $0.08
M_{\odot}$ \citet*{anIntroStellarAstro::chapter2_stellarFormation}; de no
cumplir con esta condición, el cúmulo de gas y polvo no logrará mantener la
cadena de reacciones termonucleares en su núcleo, resultando en un objeto
inerte, una \textit{enana cafe}.