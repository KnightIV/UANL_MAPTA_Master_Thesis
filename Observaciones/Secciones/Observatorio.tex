\section{Observatorio Astronómico Universitario - Iturbide}

El \textbf{Observatorio Astronómico Universitario - Iturbide} (el cual de ahora
en adelante será referido como el OAU), ubicado en el cerro Picacho en el
municipio de Iturbide, Nuevo León, es un nuevo sitio dedicado a la observación
astronómica, equipado para realizar observaciones del Sol, monitoreo de basura
espacial, y la observación de objetos variables, como los sistemas binarios o
asteroides. A continuación se describe el equipo utilizado; como software de
control se utilizó \textbf{Nighttime Imaging 'N'
Astronomy}\footnote{\url{https://nighttime-imaging.eu}} (\textbf{NINA}), el cual
permita consolidar el control de todas las componentes mecánicas en una sola
aplicación. 

% TODO: look for mount documentation
El telescopio utilizado para hacer las observaciones del sistema fue el tubo
óptico \textbf{CDK20} de \textbf{PlaneWave
Instruments}\footnote{\url{https://planewave.com/product/cdk20-ota/}}. Este
telescopio de diseño \textit{Dall-Kirkham corregido} cuenta con un grupo de
lentes frente al espejo esférico secundario, el cual resta los efectos de la
aberración esférica presente en otras configuraciones de espejos primarios y
secundarios, resultando en una imagen más nítida. Este instrumento, combinado
con una montura ecuatorial \textbf{Orion HDX110}, nos permite una vista clara de
la bóveda celeste a \ang{30} arriba del horizonte, con capacidad de observar
objetos tenues más allá de 17 magnitudes.

El CCD usado para obtener las imágenes fue el
\textbf{QHY174GPS}\footnote{\url{https://www.qhyccd.com/qhy174gps-imx174-scientific-cooled-camera/}}.
Este CCD cuenta con una resolución de $1920 \times 1200$ pixeles. Para reducir
el ruido térmico tiene un mecanismo de enfriamiento termoeléctrico, el cual lo
puede enfriar a una temperatura de -\ang{40} C bajo la temperatura ambiente. 