\section{Observatorio Astronómico Universitario - Iturbide} \label{muestra:observaciones:oau}

El \textbf{Observatorio Astronómico Universitario - Iturbide}, ubicado en el
cerro Picacho en el municipio de Iturbide, Nuevo León, es un nuevo sitio
dedicado a la observación astronómica, equipado para realizar observaciones del
Sol, monitoreo de basura espacial, y la observación de objetos variables como
sistemas binarios estelares y asteroides. A continuación se describe el equipo
utilizado; como software de control se utilizó \textbf{Nighttime Imaging 'N'
Astronomy}\footnote{\myurl{https://nighttime-imaging.eu}} (\textbf{NINA}), el
cual permita consolidar el control de todas las componentes mecánicas en una
sola aplicación. 

El telescopio utilizado para hacer las observaciones del sistema fue el tubo
óptico \textbf{CDK20} de \textbf{PlaneWave
Instruments}\footnote{\myurl{https://planewave.com/product/cdk20-ota/}} con un
número $f/6.8$. Este telescopio de diseño \textit{Dall-Kirkham corregido} cuenta
con un grupo de lentes frente al espejo esférico secundario, el cual resta los
efectos de la aberración esférica presente en otras configuraciones de espejos
primarios y secundarios, resultando en una imagen más nítida. Este instrumento,
combinado con una montura ecuatorial \textbf{Orion HDX110}, nos permite una
vista clara de la bóveda celeste a \ang{30} arriba del horizonte, con capacidad
de observar objetos tenues hasta de 17 magnitudes variando la configuración de
las exposiciones.

El CCD usado para obtener las imágenes fue el
\textbf{QHY174GPS}\footnote{\myurl{https://www.qhyccd.com/qhy174gps-imx174-scientific-cooled-camera/}}.
Este CCD cuenta con una resolución de $1920 \times 1200$ pixeles. Para reducir
el ruido térmico tiene un mecanismo de enfriamiento termoeléctrico, el cual lo
puede enfriar a una temperatura de -\ang{40} C bajo la temperatura ambiente.
Frente al CCD va equipado una rueda de filtros \textbf{ZWO
7x36mm}\footnote{\myurl{https://astronomy-imaging-camera.com/product/new-zwo-efw-7x36mm/}},
la cual puede ser equipada con 7 filtros distintos. Para las observaciones
recabadas en este trabajo, utilizamos solamente el filtro \textbf{Luminance}, el
cual representa una curva de transmisividad para la región del visible del
espectro electromagnético (\reffigure{zwoFilterTransmissionCurve}, obtenida de
la página comercial de
ZWO\footnote{\url{https://astronomy-imaging-camera.com/product/zwo-lrgb-36mm-filters/}}). 

\begin{figure}[!ht]
	\centering
	\includegraphics[scale=0.6]{Observaciones/Secciones/Figures/ZWO RGBL Transmission Curve.jpg}
	\caption{Curva de transmisión de filtros de ZWO. El filtro de
	\textbf{Luminance} se encuentra como la curva negra, abarcando todo el
	espectro visible, desde los 400nm hasta 700nm.}
	\label{zwoFilterTransmissionCurve}
\end{figure}