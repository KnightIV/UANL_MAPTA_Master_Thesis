\chapter*{Resumen}
\pagenumbering{roman}

El estudio de las estrellas es de gran importancia para entender los bloques
fundamentales que componen el universo. En particular, los \textbf{sistemas
binarios eclipsantes} son laboratorios celestes en donde se facilita el estudio
de los parámetros fundamentales de las estrellas, como sus masas y radios
absolutos, y el estudio de fenómenos físicos no observados en estrellas
solitarias, como procesos de acreción de material. Con la ayuda de censos
automatizados como \textbf{Zwicky Transient Facility} (\textbf{ZTF}) y
\textbf{Gaia} combinado con modelos computacionales sofisticados como
\textbf{PHOEBE} es posible realizar un estudio a detalle de estos sistemas,
ajustando los parámetros del modelo para explicar las variaciones presentes en
las curvas de luz observadas. 

En este trabajo se presenta el estudio del sistema binario eclipsante
\textbf{\atoObjIdNoSpace}, ubicado en las coordenadas 22h39m47.26s +45°08'47.03"
(J2000). Después de haber sido identificado como un candidato ideal a estudiar
dentro del catálogo de \textbf{Gaia} DR3 se llevo a cabo una campaña de
observación fotométrica utilizando el telescopio CDK20 de 0.5m desde el
\textbf{Observatorio Astronómico Universitario} en Iturbide, N.L., donde se
observó \atoObjId durante 9 noches. Para suplementar estas observaciones se
recopilaron curvas fotométricas del catálogo \textbf{Zwicky Transient Facility}
DR20 en las bandas ZTF:g y ZTF:r. Las tres curvas de luz indican un periodo
orbital de  $8.00558 \pm 0.00013 \ \mathrm{h}$; junto a la morfología de las
curvas de luz se plantea como hipótesis que \atoObjId es un \textbf{sistema
binario estelar en sobre-contacto}. Esta hipótesis se puso a prueba ajustando un
modelo de un sistema binario estelar eclipsante en contacto haciendo uso del
paquete de Python \textbf{PHOEBE}. Se hizo uso de varias herramientas de
estimación y optimización de parámetros en combinación con un muestreo
estadístico utilizando métodos de \textbf{Monte Carlo - Cadenas de Markov}
(\textbf{MCMC} por sus siglas en inglés) para llegar a la solución fotométrica
final presentada en la
\refthesissection{metodologia:modelocomputacional:resultados}, junto a los
diagramas de correlación entre los parámetros libres del sistema de los
parámetros fundamentales del sistema junto a una mancha estelar en la secundaria
introducida como explicación del \textbf{efecto O'Connell} observado en las
curvas de ZTF. Un análisis complementario de datos espectroscópicos junto a una
posible solución multimodal de \atoObjId y consideraciones importantes del
modelo resultante se encuentra en el
\refthesischapter{conclusion:consideraciones_phoebe}.